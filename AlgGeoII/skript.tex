\documentclass[a4paper,parskip=half,numbers=enddot, DIV=12]{scrreprt}
\usepackage[utf8]{inputenc}

\usepackage{../header}
\usepackage{../frankenumbering}
\usepackage{../shortcuts}

\usepackage{csquotes}
%\usepackage{tikz-cd}%I cannot draw diagrams without it - Felix. %well, I can - Ferdinand
\usepackage[backend=biber,style=numeric,sorting=none]{biblatex}
\setcounter{biburlnumpenalty}{7000}
\setcounter{biburllcpenalty}{7000}
\setcounter{biburlucpenalty}{8000}
\addbibresource{../literatur.bib}

% Title Page
\title{Algebraic Geometry II}
\author{Ferdinand Wagner}
\date{Sommersemester 2018}

\displaywidowpenalty=8000
%\postdisplaypenalty=8000
\widowpenalty=8000
\clubpenalty=8000

\begin{document}
\pagenumbering{Alph}
\maketitle
\pagenumbering{roman}

\thispagestyle{plain}
This text consists of notes on the lecture Algebraic Geometry II, taught at the University of Bonn by Professor Jens Franke in the summer term (Sommersemester) 2018. 

Please report bugs, typos etc. through the \emph{Issues} feature of github.

\tableofcontents

\addchap{Introduction}
\pagenumbering{arabic}
This lecture will develop the cohomology of (quasi)coherent sheaves of modules. Professor Franke assumes familiarity with the contents of last term's Algebraic Geometry I. In particular, this includes the category of (pre)schemes, equalizers and fibre products of preschemes as well as in arbitrary categories and quasi-coherent $\Oo_X$-modules. If you are want to brush up your knowledge about these topics, the \href{https://github.com/Nicholas42/AlgebraFranke/tree/master/AlgGeoI}{\emph{lecture notes from Algebraic Geometry I}} \cite{alggeo1} might be your friend.

Professor Franke started the lecture with an example of sheaf cohomology entering the game. Let $X$ be a topological space, $\Cc_X$ the sheaf of continuous $\IC$-valued functions on $X$ and $\underline{\IZ}_X$ the sheaf of locally constant (i.e., continuous) functions on $X$ with values in $\IZ$. Then there is a short exact sequence
\begin{align*}
	0\morphism \underline{\IZ}_X\morphism[\cdot 2\pi\mathrm{i}] \Cc_X\morphism[\exp]\Cc_X^\times\morphism 0
\end{align*}
of sheaves of abelian groups. In general, taking global section doesn't preserve exactness but gives rise to a long exact sequence
\begin{diagram*}
	\node[ob] (0o) at (0,1.5) {$0$};
	\node[ob] (0u) at (0,0) {$0$};
	\node[ob] (H0Z) [right=0.5 of 0u] {$H^0(X,\underline{\IZ}_X)$};
	\node[ob] (H0CX) [right=0.5 of H0Z] {$H^0(X,\Cc_X)$};
	\node[ob] (H0CXx) [right=0.5 of H0CX] {$H^0(X,\Cc_X^\times)$};
	\node[ob] (Z) at (0o -| H0Z) {$\underline{\IZ}_X(X)$};
	\node[ob] (CX) at (0o -| H0CX) {$\Cc_X(X)$};
	\node[ob] (CXx) at (0o -| H0CXx) {$\Cc_X^\times(X)$};
	\node[ob, shift={(0,1.5)}] (H1Z) [right=0.75 of H0CXx] {$H^1(X,\underline{\IZ}_X)$};
	\node[ob] (H1CX) [right=0.5 of H1Z] {$H^1(X,\Cc_X)$};
	\node[ob] (dots) [right=0.5 of H1CX] {$\ldots$};
	\scriptsize
	\draw[->] (0o) -- (Z);
	\draw[->] (0u) -- (H0Z);
	\draw[->] (Z) -- (CX);
	\draw[->] (H0Z) -- (H0CX);
	\draw[->] (CX) -- (CXx);
	\draw[->] (H0CX) -- (H0CXx);
	\draw[->] (CXx) -- (H1Z) node[pos=0.5, above] {$d$};
	\draw[->] (H0CXx) -- (H1Z);
	\draw[->] (H1Z) -- (H1CX);
	\draw[->] (Z) -- (H0Z) node[pos=0.5, above=-0.25ex, sloped] {$\sim$};
	\draw[->] (CX) -- (H0CX) node[pos=0.5, above=-0.25ex, sloped] {$\sim$};
	\draw[->] (CXx) -- (H0CXx) node[pos=0.5, above=-0.25ex, sloped] {$\sim$};
	\draw[->] (H1CX) -- (dots);
\end{diagram*}
in which the $H^k(X,\underline{\IZ}_X)$, $H^k(X,\Cc_X)$, and $H^k(X,\Cc_X^\times)$ are \emph{sheaf cohomology groups}. There is the more general notion of \emph{derived functors} (Grothendieck, T\^{o}hoku paper), but this won't appear in the lecture.

Background in homological algebra is not required safe for cohomology groups of cochain complexes, the long exact cohomology sequence and the following famous lemma.
\begin{lem*}[Five lemma] Given a diagram
	\begin{diagram*}
		\node[ob] (A) at (0,1.5) {$A$};
		\node[ob] (B) at (1.5,1.5) {$B$};
		\node[ob] (C) at (3,1.5) {$C$};
		\node[ob] (D) at (4.5,1.5) {$D$};
		\node[ob] (E) at (6,1.5) {$E$};
		\node[ob] (A') at (0,0) {$A'$};
		\node[ob] (B') at (1.5,0) {$B'$};
		\node[ob] (C') at (3,0) {$C'$};
		\node[ob] (D') at (4.5,0) {$D'$};
		\node[ob] (E') at (6,0) {$E'$};
		\scriptsize
		\draw[->] (A) -- (B);
		\draw[->] (B) -- (C);
		\draw[->] (C) -- (D);
		\draw[->] (D) -- (E);
		\draw[->] (A') -- (B');
		\draw[->] (B') -- (C');
		\draw[->] (C') -- (D');
		\draw[->] (D') -- (E');
		\draw[->] (A) -- (A') node[pos=0.5, left] {$\alpha$};
		\draw[->] (B) -- (B') node[pos=0.5, left] {$\beta$};
		\draw[->] (C) -- (C') node[pos=0.5, left] {$\gamma$};
		\draw[->] (D) -- (D') node[pos=0.5, left] {$\delta$};
		\draw[->] (E) -- (E') node[pos=0.5, left] {$\epsilon$};
	\end{diagram*}
	of (abelian) groups/$R$-modules/etc.\ with exact rows, in which $\alpha$, $\beta$, $\delta$, and $\epsilon$ are isomorphisms, then $\gamma$ is an isomorphism as well.
\end{lem*}
\begin{proof}
	Easy diagram chase.
\end{proof}

\chapter{Cohomology of quasi-coherent sheaves of modules}
\section{Recollection of basic definitions and results}
\begin{defi}[{\cite[Definition~1.5.2 and Definition~1.5.9\itememph{b}]{alggeo1}}]
	\begin{alphanumerate}
		\item A \defemph{prescheme} (Franke uses ``EGA termology'') is a locally ringed space $(X,\Oo_X)$ which locally has the form $\Spec R$ for some rings $R$.
		\item A prescheme $X$ is called a \defemph{scheme}, if, for any prescheme $T$ and any pair of morphisms $T\doublemorphism[a][b]X$, the equalizer $\Eq\Big(T\doublemorphism[a][b]X\Big)$ is a closed subprescheme of $X$.
	\end{alphanumerate}
\end{defi}
\begin{rem*}
	Equivalently, a prescheme $X$ is a scheme iff the diagonal $\Delta\colon X\xrightarrow{(\id_X,\id_X)}X\times X$ is a closed immersion (cf.\ \cite[Fact~1.5.8]{alggeo1}). In other words, schemes are \emph{separated} preschemes
\end{rem*}
\begin{prop}
	If $U$ and $V$ are affine open subsets of a scheme $X$, then their intersection $U\cap V$ is again affine (and open of course).
\end{prop}
\begin{proof}
	This was proved in \cite[Proposition~1.5.4]{alggeo1}.
\end{proof}


\appendix
\chapter{Appendix -- category theory corner}
\setcounter{thm}{0}
\renewcommand*{\thethm}{\Alph{thm}}
\section{Towards abelian categories}
\begin{defi}
	\begin{alphanumerate}
		\item \lbl{def:additiveCategory}A \defemph{pointed} category is a category with initial and final objects, such that the canonical (unique) morphism from the initial to the final object is an isomorphism.
		\item An \defemph{additive} category $\Aa$ is a pointed category which has a product $X\times Y$ (i.e., a fibre product over the final object $*$) and coproduct $X\amalg Y$ (i.e., a dual fibre product with respect to the initial object $*$) such that the canonical morphism $X\amalg Y\morphism X\times Y$ is an isomorphism for all objects $X,Y\in\Ob(\Aa)$ and such that the resulting addition law on $\Hom_\Aa(X,Y)$ defines a group structure for all $X,Y\in\Ob(\Aa)$.
	\end{alphanumerate}
\end{defi}
\begin{rem*}
	 \begin{alphanumerate}
	 	\item When $\Aa$ is a pointed category and $X,Y\in\Ob(\Aa)$, let the \emph{zero morphism} (which we denote $0$) $X\morphism[0]Y$ be defined by $X\morphism *\morphism Y$, where $*$ is the both initial and final object.
	 	\item We will construct the canonical morphism $X\amalg Y\morphism[c]X\times Y$ from Definition~\reff{def:additiveCategory}\itememph{b}. The product $X\times Y$ comes with canonical projections $X\lmorphism[p_1]X\times Y\morphism[p_2]Y$ such that given morphisms $T\morphism[\xi]X$ and $T\morphism[\upsilon]Y$ there is a unique $T\morphism[\xi\times\upsilon]X\times Y$ such that
	 	\begin{diagram*}
	 		\node (XY) at (0,0) {$X\times Y$};
	 		\node (X) at (-1,1.25) {$X$};
	 		\node (Y) at (-1,-1.25) {$Y$};
	 		\node (T) at (2.5,0) {$T$};
	 		\scriptsize
	 		\draw[->] (XY) -- (X) node[pos=0.5, above right] {$p_1$};
	 		\draw[->] (XY) -- (Y) node[pos=0.5, below right] {$p_2$};
	 		\draw[->, dashed] (T) -- (XY) node[pos = 0.5, above] {$\exists!\ \xi\times \upsilon$};
	 		\draw[->, bend right] (T) to node[pos=0.5,below left] {$\xi$} (X);
	 		\draw[->, bend left] (T) to node[pos=0.5,above left] {$\upsilon$} (Y);
	 	\end{diagram*}
	 	commutes.
	 	
	 	Similarly, the coproduct $X\amalg Y$ has morphisms $X\morphism[i_1]X\amalg Y\lmorphism[i_2]Y$ such that given morphisms $X\morphism[\xi]T$ and $Y\morphism[\upsilon]T$ there is a unique morphism $X\amalg Y\morphism[\xi\amalg\upsilon]T$ such that
	 	\begin{diagram*}
	 		\node (XY) at (0,0) {$X\amalg Y$};
	 		\node (X) at (-1,1.25) {$X$};
	 		\node (Y) at (-1,-1.25) {$Y$};
	 		\node (T) at (2.5,0) {$T$};
	 		\scriptsize
	 		\draw[<-] (XY) -- (X) node[pos=0.5, above right] {$i_1$};
	 		\draw[<-] (XY) -- (Y) node[pos=0.5, below right] {$i_2$};
	 		\draw[->, dashed] (XY) -- (T) node[pos = 0.5, above] {$\exists!\ \xi\amalg \upsilon$};
	 		\draw[<-, bend right] (T) to node[pos=0.5,below left] {$\xi$} (X);
	 		\draw[<-, bend left] (T) to node[pos=0.5,above left] {$\upsilon$} (Y);
	 	\end{diagram*}
	 	commutes.
	 	
	 	Using the universal property of $X\times Y$, we get a unique morphism $X\morphism[\alpha]X\times Y$ such that $p_1\alpha=\id_X$, $p_2\alpha=0$ and a unique morphism $Y\morphism[\beta]X\times Y$ such that $p_1\beta=0$ and $p_2\beta=\id_Y$. Then 
	 	\begin{align*}
	 		c\colon X\amalg Y\xrightarrow{\alpha\amalg\beta}X\times Y
	 	\end{align*}
	 	is the morphism we are looking for. It is unique with the property that $p_1 c i_1=\id_X$, $p_1 c i_2=0$, $p_2 c i_1=0$, and $p_2 c i_2=\id_Y$.
	 	\item For abelian groups and modules over a ring, both $X\amalg Y$ and $X\times Y$ are given by $\left\{(x,y)\st x\in X,\ y\in Y\right\}$ with component-wise operations and $p_1(x,y)=x$, $p_2(x,y)=y$, $i_1(x)=(x,0)$, and $i_2(y)=(0,y)$.
	 	\item For an additive category $\Aa$, it follows that finite products $\prod_{i=1}^nX_i$ and coproducts $\coprod_{i=1}^nX_i$ (of some objects $X_1,\ldots,X_n\in\Ob(\Aa)$) exist and are canonically isomorphic. We typically denote both by $\bigoplus_{i=1}^nX_i$ in that case.
	 	\item We would like to describe the addition on $\Hom_\Aa(X,Y)$. For a pair of morphisms $X\doublemorphism[a][b]Y$ we denote the composition
	 	\begin{align*}
	 		X\xrightarrow{\id_X\times\id_X}X\oplus X\xrightarrow{a\amalg b}Y
	 	\end{align*}
	 	by $a+b$. Then $0$ is a neutral element and associativity holds, but the existence of inverse elements needs to be imposed to obtain indeed a group structure.
	 	\item It is, however, automatically abelian. What we need to show is $(a\amalg b)\circ\Delta=(b\amalg a)\circ\Delta$ with $\Delta=\id_X\times \id_X$. The universal property of coproducts gives a unique $X\oplus X\morphism[\sigma]X\oplus X$ such that
	 	\begin{diagram*}
	 		\node (XX2) at (0,0) {$X\oplus X$};
	 		\node (X1) at (-1,1.25) {$X$};
	 		\node (X2) at (-1,-1.25) {$X$};
	 		\node (XX1) at (-2,0) {$X\oplus X$};
	 		\node (Y) at (2.5,0) {$Y$};
	 		\scriptsize
	 		\draw[<-] (XX2) -- (X1) node[pos=0.5, above right] {$i_1$};
	 		\draw[<-] (XX2) -- (X2) node[pos=0.5, below right] {$i_2$};
	 		\draw[->] (XX2) -- (Y) node[pos = 0.5, above] {$a\amalg b$};
	 		\draw[<-, bend right] (Y) to node[pos=0.5,below left] {$a$} (X1);
	 		\draw[<-, bend left] (Y) to node[pos=0.5,above left] {$b$} (X2);
	 		\draw[<-] (XX1) -- (X1) node[pos=0.5, above left] {$i_2$};
	 		\draw[<-] (XX1) -- (X2) node[pos=0.5, below left] {$i_1$};
	 		\draw[->, dashed] (XX1) -- (XX2) node[pos=0.5, above] {$\exists!\ \sigma$};
	 	\end{diagram*}
	 	commutes. Then $\sigma$ is easily seen to be an isomorphism and $b\amalg a=(a\amalg b)\circ \sigma$ by the uniqueness of $b\amalg a$. It thus suffices to show $\sigma\Delta=\Delta$. By the uniqueness of $\Delta$, this is equivalent to $p_1\sigma\Delta=\id_X$ and $p_2\sigma\Delta=\id_X$. We claim that $p_1\sigma=p_2$ and vice versa, which would finish the proof. To see this, note that $p_1\sigma=p_2$ is equivalent to $p_1\sigma i_1=p_2i_1=0$ and $p_1\sigma i_2=p_2i_2=\id_X$ by the universal property of the coproduct $X\oplus X$. This follows from $\sigma i_1=i_2$ and $\sigma i_2=i_1$ by definition of $\sigma$.
	 \end{alphanumerate}
\end{rem*}
\begin{example*} The following are additive categories.
	\begin{alphanumerate}
		\item Modules over a given ring $R$ (in particular, abelian groups).
		\item Sheaves of modules.
		\item Banach spaces with bounded linear maps as morphisms. The common initial and final object is the zero space and $A\oplus B=\left\{(a,b)\st a\in A,\ b\in B\right\} $ with $\max\{\|a\|,\|b\|\}$ or $\|a\|+\|b\|$ as norm (this category will turn out not to be abelian).
		\item Free or projective modules over a ring $R$.
	\end{alphanumerate}
\end{example*}
\begin{rem}
	For kernels and cokernels in an additive category $\Aa$ the following universal properties are imposed. The kernel $\ker(A\morphism[\alpha]B)$ must come with a morphism $\ker(\alpha)\morphism[\iota]A$ and satisfy
	\begin{align*}
		\Hom_\Aa\left(T,\ker\Big(A\morphism[\alpha]B\Big)\right)&\isomorphism \left\{f\in \Hom_\Aa\st \alpha f=0\right\}\\
		\left(T\morphism[\tau]A\right)&\longmapsto f=\iota\tau
	\end{align*}
	for any test object $T\in\Ob(\Aa)$. Similarly, cokernels come with a morphism $B\morphism[\pi]\coker(\alpha)$ and satisfy
	\begin{align*}
		\Hom_\Aa\left(\coker\Big(A\morphism[\alpha]B\Big),T\right)&\isomorphism \left\{g\in\Hom_\Aa(B,T)\st g\alpha=0\right\}\\
		\left(\coker(\alpha)\morphism[\tau]T\right)&\longmapsto g=\tau\pi
	\end{align*}
	for any test object $T\in\Ob(\Aa)$.
	
	Thus, 
	\begin{align*}
		\ker\Big(A\morphism[\alpha]B\Big)=\Eq\Big(A\doublemorphism[\alpha][0]B\Big)\quad\text{and}\quad\coker\Big(A\morphism[\alpha]B\Big)=\Coeq\Big(A\doublemorphism[\alpha][0]B\Big)\;. 
	\end{align*}
	For abelian categories, the existence of kernels and cokernels is required, in addition to additivity. Equivalent conditions are the existence of equalizers and coequalizers, fibre products and dual fibre products, or the existence of finite limits and colimits as
	\begin{align*}
		\Eq\Big(A\doublemorphism[\alpha][\beta]B\Big)=\ker\Big(A\xrightarrow{\alpha-\beta}B\Big)\quad\text{and}\quad \Coeq\Big(A\doublemorphism[\alpha][\beta]B\Big)=\coker\Big(A\xrightarrow{\alpha-\beta}B\Big)
	\end{align*}
	(the minus here is the one obtained from additivity of $\Aa$).
	
	Recall that a morphism $A\morphism[i]B$ is an \emph{effective monomorphism}, if the following equivalent conditions hold.
	\begin{alphanumerate}
		\item (In any category) We have a bijection
		\begin{align*}
			\Hom_\Aa(T,A)&\isomorphism\left\{f\in\Hom_\Aa(T,B)\st 
			\begin{array}{c}
				\alpha f=\beta f\text{ if }B\doublemorphism[\alpha][\beta]S\text{ is any pair of}\\ 
				\text{morphisms such that }\alpha i=\beta i
			\end{array}
			\right\}\\
			t\in\Hom_\Aa(T,A) &\longmapsto f=it\;.
		\end{align*}
		\item (If the category has finite colimits) $i$ is an equalizer of something.
		\item (In additive categories with kernels and cokernels) $i$ is the kernel of an appropriate morphism
		\item (In additive categories with kernels and cokernels) $i$ is the kernel of its cokernel.
	\end{alphanumerate}
	Dually, $A\morphism[p]B$ is an \emph{effective epimorphism} if the following equivalent conditions hold.
	\begin{alphanumerate}
		\item (In any category) We have a bijection
		\begin{align*}
			\Hom_\Aa(B,T)&\isomorphism\left\{f\in\Hom_\Aa(A,T)\st 
			\begin{array}{c}
				f\alpha =f\beta\text{ if }S\doublemorphism[\alpha][\beta]A\text{ is any pair of}\\
				\text{morphisms such that }p\alpha =p\beta 
			\end{array}
			\right\}\\
			t\in\Hom_\Aa(B,T) &\longmapsto f=tp\;.
		\end{align*}
		\item (If the category has finite limits) $p$ is a coequalizer of something.
		\item (In additive categories with kernels and cokernels) $p$ is the cokernel of an appropriate morphism
		\item (In additive categories with kernels and cokernels) $p$ is the cokernel of its kernel.
		\item $B^\op\morphism[p^\op]A^\op$ is an effective monomorphism in the dual category $\Aa^\op$.
	\end{alphanumerate}
	In any category, a morphism which is mono and effectively epi (or epi and effectively mono) is an isomorphism, but there are examples of morphisms which are simultaneously mono and epi but not an isomorphism (e.g.\ $\IZ\monomorphism\IQ$ in the category of rings). This needs to be ruled out by a definition, and that's what is happening now! 
\end{rem}
\begin{defi}
	A category $\Aa$ is \defemph{abelian}, if it is additive, has kernels and cokernels and such that every monomorphism is effectively mono, every epimorphism is effectively epi, and (thus) any morphism which is both a mono- and an epimorphism is an isomorphism.
\end{defi}
The category of modules (over a ring $R$) or sheaves of modules are abelian categories, but not Banach spaces or projective modules over most rings.
\begin{example}
	We show that the category $\Rr\cat{-Mod}$ of sheaves of modules (over a sheaf of rings $\Rr$ on some topological space $X$) is abelian.
	
	\begin{proof}
	\emph{Step 1.} We verify that $\Rr\cat{-Mod}$ is additive. First note that the zero sheaf $0$ is a common initial and final object. A direct sum of $\Mm,\Nn\in\Ob(\Rr\cat{-Mod})$ is given by
	\begin{align*}
		(\Mm\oplus\Nn)(U)=\left\{(m,n)\st m\in\Mm(U), n\in\Nn(U)\right\}\quad\text{for all }U\subseteq X\text{ open}
	\end{align*}
	(it's clear that this is a presheaf and it inherits the sheaf axiom from $\Aa$ and $\Nn$) with component-wise module operations and with $\Mm\lmorphism[p]\Mm\oplus\Nn\morphism[q]\Nn$ and $\Mm\morphism[i]\Mm\oplus\Nn\lmorphism[j]\Nn$ given by $p(m,n)=m$, $q(m,n)=n$, $i(m)=(m,0)$, and $j(n)=(0,n)$ on open subsets $U\subseteq X$ and $m\in\Mm(U)$, $n\in\Nn(U)$.
	
	If $\Mm\morphism[\mu]\Tt\lmorphism[\nu]\Nn$ are given, $\Mm\oplus\Nn\morphism[\mu\amalg\nu]\Tt$ sending $(m,n)\in(\Mm\oplus\Nn)(U)$ to $\mu(m)+\nu(n)$ verifies the universal property of the coproduct for $\Mm\oplus\Nn$. Similarly, $\Tt\morphism[\mu\times\nu]\Mm\oplus\Nn$ given by $(\mu\times\nu)(t)=(\mu(t),\nu(t))$ for $t\in\Tt(U)$ confirms the universal property of the product for $\Mm\oplus\Nn$. Also, $c=\id_{\Mm\oplus\Nn}$ is the unique endomorphism $c$ of that object such that $pci=\id_\Mm$, $qcj=\id_\Nn$, $pcj=0$, and $qci=0$. Thus, $\Rr\cat{-Mod}$ is additive (the group laws being easily verified).
	
	\emph{Step 2.} Now about \emph{kernels}. Let $\Mm\morphism[f]\Nn$ be a morphism of sheaves of $\Rr$-modules and $\Kk$ be the sheaf given by 
	\begin{align*}
		\Kk(U)=\ker\left(\Mm\morphism[f]\Nn\right)(U)\coloneqq\ker\left(\Mm(U)\morphism[f]\Nn(U)\right)
	\end{align*}
	(you should convince yourself that this indeed satisfies the sheaf axiom). Then the inclusion $\Kk\morphism[\kappa]\Mm$ is a monomorphism as $\Kk(U)\monomorphism\Mm(U)$ is injective for every open subset $U\subseteq X$. If $\Tt\morphism[\tau]\Mm$ is a morphism such that $f\tau=0$, then for every $t\in\Tt(U)$ we have $f(\tau(t))=0$, hence $\snake{\tau}(t)\coloneqq\tau(t)\in\ker\left(\Mm(U)\morphism[f]\Nn(U)\right)=\Kk(U)$ and $\tau$ factors over 
	\begin{diagram*}
		\node[ob](R) at (0,1.25) {$\Tt$};
		\node[ob](A) at (2.5,1.25) {$\Mm$};
		\node[ob](RS) at (1.25,0) {$\Kk$};
		\scriptsize
		\draw[->] (R) -- (A) node[pos=0.5, above] {$\tau$};
		\draw[->, dashed] (R) -- (RS) node[pos=0.5, below left] {$\exists!\ \snake\tau$};
		\draw[right hook->] (RS) -- (A) node[pos=0.5, below right] {$\kappa$};
	\end{diagram*}
	This proves that $\Kk$ is indeed a kernel of $f$ in the category $\Rr\cat{-Mod}$.
	
	It is a consequence of the exactness of the $\colimit$ functor, that 
	\begin{align*}
		\Kk_x\simeq\ker\left(\Mm_x\morphism[f]\Nn_x\right)\;.
	\end{align*}
	
	More generally, one may check that in any additive category (with kernels), a morphism $i$ is a monomorphism iff $\ker(i)=0$. Thus, in our example we have the equivalent conditions
	\begin{alphanumerate}
		\item $\Mm\morphism[f]\Nn$ is a monomorphism
		\item $\Mm(U)\morphism[f]\Nn(U)$ is injective for all open subsets $U\subseteq X$
		\item $\ker(f)=0$ (the zero sheaf)
		\item $\Mm_x\morphism[f]\Nn_x$ is injective for all $x\in X$.
	\end{alphanumerate}
	
	\emph{Step 3.} We construct cokernels, which won't be that straightforward (duh!), related to the fact that epimorphisms aren't that simple in categories of sheaves. If $f$ is such that $\Mm(U)\morphism[f]\Nn(U)$ is surjective for all open $U$, then $f$ is an epimorphism, but there are epimorphisms $f$ for which this fails. It follows from the fact that a sheaf $\Gg$ is canonically isomorphic to its sheafification $\Gg^\sh$ (cf. \cite[Proposition~1.2.1\itememph{d}]{alggeo1}) that a morphism between sheaves (of sets, groups, \ldots) is uniquely determined by the maps it induces on stalks. Thus, $\Gg\morphism\Hh$ is an epimorphism when $\Gg_x\morphism\Hh_x$ is an epimorphism in the respective target category , for all $x\in X$.
	
	Now for the cokernel construction. For a morphism $\Mm\morphism[f]\Nn$ of sheaves of $\Rr$-modules, we put
	\begin{align*}
		\coker(f)=\Cc\coloneqq\left(U\mapsto \coker\left(\Mm(U)\morphism[f]\Nn(U)\right)\right)^\sh
	\end{align*}
	Note that $\Cc_x\morphism\coker\left(\Mm_x\morphism[f]\Nn_x\right)$ sending the image of $\nu=(\nu_y)_{y\in U}\in\Cc(U)$ under $\Cc(U)\morphism\Cc_x$ to $\nu_x$ is an isomorphism. If $\nu_x=0$, by the coherence condition there is some $n\in\Nn(V)$ such that $\nu_y=\left(\text{image of }n\text{ in }\coker\left(\Mm_y\morphism[f]\Nn_y\right)\right)$ for all $y\in U\cap V$ and such that $x\in V$. Replacing $U$ by $U\cap V$ and $\nu$ by $\nu|_{U\cap V}$, we may assume that $U=V$. The image of $n$ in $\Nn_x$ is in the image of $\Mm_x\morphism[f]\Nn_x$ as its image in $\coker\left(\Mm_x\morphism[f]\Nn_x\right)$ is $\nu_x=0$. By the definition of stalks, it follows that there are an open neighbourhood $x\in V\subseteq U$ and $m\in\Mm(U)$ such that $n|_V=f(m)$. But then $\nu_y$ vanishes when $y\in V$, hence $\nu|_V=0$, showing that $\Cc_x\morphism\coker\left(\Mm_x\morphism[f]\Nn_x\right)$ is injective.
	
	If $\upsilon\in\coker\left(\Mm_x\morphism[f]\Nn_x\right)$ is the image of some $\xi\in\Nn_x$ which is the image of some $n\in\Nn(U)$ and $\upsilon$ is the image of
	\begin{align*}
		\left(\text{image of }n\text{ under }\Nn(U)\morphism\Nn_y\morphism\coker\left(\Mm_y\morphism[f]\Nn_y\right)\right)_{y\in U}\in\Cc(U)\;.
	\end{align*}
	Hence $\Cc_x\morphism\coker\left(\Mm_x\morphism[f]\Nn_x\right)$ is surjective.
	
	We have a morphism $\Nn\morphism\Cc$ sending $n\in \Nn(U)$ to
	\begin{align*}
		\left(\text{image of }n\text{ under }\Nn(U)\morphism\Nn_x\morphism\coker\left(\Mm_x\morphism[f]\Nn_x\right)\right)_{x\in U}\;.
	\end{align*}
	By the previous calculation $\Cc_x\simeq\coker\left(\Mm_x\morphism[f]\Nn_x\right)$, this morphism $\Nn\morphism\Cc$ induces surjections on stalks, hence is an epimorphism of sheaves. We show that the morphism $\Nn\morphism\Cc$ satisfies the universal property of the cokernel.
	
	Let $\Nn\morphism[\tau]\Tt$ be a morphism of sheaves of $\Rr$-modules such that $\tau f=0$. When $\nu=(\nu_x)_{x\in U}\in\Cc(U)\subseteq\prod_{x\in U}\coker\left(\Mm_x\morphism[f]\Nn_x\right)$ we define $\tau_1(\nu)\in\prod_{x\in U}\Tt_x$ by selecting $n\in\Nn_x$ whose image in $\coker\left(\Mm_x\morphism[f]\Nn_x\right)$ equals $\nu_x$, then put $\tau_1(\nu)_x=\tau(n)_x$ which is independent of the choice of $n$. It follows from the coherence condition for $\Cc$ that $\tau_1(\nu)\in\Tt^\sh(U)\subseteq\prod_{x\in U}\Tt_x$. Hence there is $\Cc\morphism[\tau_2]\Tt$ such that $\tau_1=\left(\Tt\isomorphism\Tt^\sh\right)\circ\tau_2$ and $\tau_2$ makes
	\begin{diagram*}
		\node[ob](R) at (0,1.25) {$\Nn$};
		\node[ob](A) at (2.5,1.25) {$\Tt$};
		\node[ob](RS) at (1.25,0) {$\Cc$};
		\scriptsize
		\draw[->] (R) -- (A) node[pos=0.5, above] {$\tau$};
		\draw[->, dashed] (R) -- (RS) node[pos=0.5, below left] {$\exists!\ \tau_2$};
		\draw[->] (RS) -- (A);
	\end{diagram*}
	commutative (uniqueness of $\tau_2$ is easy to see stalk-wise). It follows that $\Nn\morphism\Cc$ is ineed a cokernel of $f$.
	
	More generally, one may check that in any additive category (with cokernels) a morphism $f$ is an epimorphism if $\coker(f)=0$. By our previous construction of cokernels and the description of stalks, we have equivalent conditions
	\begin{alphanumerate}
		\item $\Mm\morphism[f]\Nn$ is an epimorphism of sheaves of $\Rr$-modules.
		\item $\Mm_x\morphism[f]\Nn_x$ is surjective for all $x\in X$.
		\item For every open $U\subseteq X$ and $n\in\Nn(U)$ there are an open covering $U=\bigcup_{\lambda\in\Lambda}U_\lambda$ and $m_\lambda\in\Mm(U_\lambda)$ such that $n|_{U_\lambda}=f(m_\lambda)$
	\end{alphanumerate}
	\ldots but \itememph{c} does \emph{not} imply the surjectivity of $\Mm(U)\morphism[f]\Nn(U)$, unless, e.g., $f$ is also a monomorphism.
	
	\emph{Step 4.} We verify the rest of the abelianness conditions. First, let $\Mm\morphism[f]\Nn$ be a mono- and epimorphism. Then it induces isomorphisms on stalks, hence is an isomorphism itself.
	
	Let $\Mm\morphism[i]\Nn$ be a monomorphism and $\Nn\morphism\Cc$ be its cokernel. Then $\ker\left(\Nn\morphism\Cc\right)_x=\ker\left(\Nn_x\morphism\Cc_x\right)=\ker\left(\Nn_x\morphism\coker\left(\Mm_x\morphism[i]\Nn_x\right)\right)\simeq\Mm_x$ as $\Mm_x\morphism[i]\Nn_x$ is injective. Hence $\Mm\morphism\ker\left(\Nn\morphism\Cc\right)$ induces isomorphisms on stalks and thus is an isomorphism itself. It follows that any monomorphism is an effective monomorphism.
	
	Similar arguments apply to epimorphisms.	
	\end{proof}
\end{example}

\printbibliography

\end{document}          

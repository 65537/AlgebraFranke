\documentclass[a4paper,parskip=half,numbers=enddot, DIV=12]{scrreprt}
\usepackage[utf8]{inputenc}

\usepackage{../header}
\usepackage{../frankenumbering}
\usepackage{../shortcuts}

\usepackage{csquotes}
%\usepackage{tikz-cd}%I cannot draw diagrams without it - Felix. %well, I can - Ferdinand
\usepackage[backend=biber,style=numeric,sorting=none]{biblatex}
\setcounter{biburlnumpenalty}{7000}
\setcounter{biburllcpenalty}{7000}
\setcounter{biburlucpenalty}{8000}
\addbibresource{../literatur.bib}

% Title Page
\title{Homological Methods in Commutative Algebra}
\author{Ferdinand Wagner}
\date{Sommersemester 2018}

\displaywidowpenalty=8000
%\postdisplaypenalty=8000
\widowpenalty=8000
\clubpenalty=8000

\begin{document}
\pagenumbering{Alph}
\maketitle
\pagenumbering{roman}

\thispagestyle{plain}
This text consists of notes on the lecture Homological Methods in Commutative Algebra, taught at the University of Bonn by Professor Jens Franke in the summer term (Sommersemester) 2018. 

Please report bugs, typos etc. through the \emph{Issues} feature of github.

\tableofcontents

\addchap{Introduction}
\pagenumbering{arabic}

Professor Franke started the lecture giving an idea of what the $\Tor$ and $\Ext$ functors do. Let $R$ be a commutative ring with $1$. For an exact sequence of $R$-modules
\begin{align*}
	0\morphism M'\morphism M\morphism M''\morphism 0
\end{align*}
and $T$ another $R$-module, the sequence 
\begin{align}\lbl{eq:TensorSequence}
	M'\otimes_RT\morphism M\otimes_RT\morphism M''\otimes_RT\morphism 0
\end{align}
is exact but usually can't be extended by $0$ on the left end. The same is true for
\begin{align}\lbl{eq:HomSequence}
	0\morphism\Hom_R(T,M')\morphism\Hom_R(T,M)\morphism\Hom_R(T,M'')
\end{align}
and
\begin{align}\lbl{eq:HomSequence2}
	0\morphism\Hom_R(M'',T)\morphism\Hom_R(M,T)\morphism\Hom_R(M',T)\;,
\end{align}
but again, they can't be extended by $0$ on the right in general.
\begin{example*}
	Take $R=\IZ$ and consider the exact sequence $0\morphism\IZ\morphism[\cdot2]\IZ\morphism\IZ/2\IZ\morphism0$. 
	\begin{alphanumerate}
		\item Let $T=\IZ/2\IZ$ in \eqreff{eq:TensorSequence}. Then $\IZ\otimes_\IZ\IZ/2\IZ\simeq\IZ/2\IZ$ and $\IZ/2\IZ\morphism[\cdot 2]\IZ/2\IZ$ is the zero morphism, showing that injectivity on the left end fails in \eqreff{eq:TensorSequence}.
		\item Let $T=\IZ/2\IZ$ in \eqreff{eq:HomSequence}. We claim that surjectivity fails on the right end. Indeed, if it was surjective, then $\id_{\IZ/2\IZ}\in\Hom_\IZ(\IZ/2\IZ,\IZ/2\IZ)$ would have to have a lift
		\begin{diagram*}
			\node[ob] (Z2) at (0,0) {$\IZ/2\IZ$};			
			\node[ob] (Z22) at (1.75,0) {$\IZ/2\IZ$};
			\node[ob] (Z) at (1.75,1.5) {$\IZ$};
			\draw[transform canvas={yshift=1pt}] (Z2) -- (Z22);
			\draw[transform canvas={yshift=-1pt}] (Z2) -- (Z22);
			\draw[->>] (Z) -- (Z22);
			\draw[->, dashed] (Z2) -- (Z);
		\end{diagram*}
		which it hasn't as $\IZ$ is $2$-torsion free and thus every morphism $\IZ/2\IZ\morphism\IZ$ must be $0$.
		\item Let $T=\IZ$ in \eqreff{eq:HomSequence2}. We claim that that surjectivity fails on the right end, or more specifically, that $\id_{\IZ}\in\Hom_\IZ(\IZ,\IZ)$ has no preimage. Indeed, if $f\in\Hom_\IZ(\IZ,\IZ)$ is a preimage of $\id_\IZ$, i.e. the composition $\IZ\morphism[\cdot2]\IZ\morphism[f]\IZ$ equals $\id_\IZ$, then $f$ must be given by $f(n)=\frac{n}{2}$ on $2\IZ$, but this can't be extended to all of $\IZ$, contradiction!
	\end{alphanumerate}
\end{example*}
To handle this deficiency, one constructs \emph{derived functors} $\Tor$ and $\Ext$, which give rise to long exact sequences
\begin{multline*}
	\ldots\morphism\Tor_2^R(M'',T)\morphism \Tor_1^R(M',T)\morphism \Tor_1^R(M,T)\morphism \Tor_1^R(M'',T)\\
	\morphism M'\otimes_RT\morphism M\otimes_RT\morphism M''\otimes_RT\morphism 0\;,
\end{multline*}
as well as
\begin{multline*}
	0\morphism\Hom_R(T,M')\morphism\Hom_R(T,M)\morphism\Hom_R(T,M'')\\
	\morphism\Ext_R^1(T,M')\morphism\Ext_R^1(T,M)\morphism\Ext_R^1(T,M'')\morphism\Ext_R^2(T,M')\morphism\ldots
\end{multline*}
and
\begin{multline*}
0\morphism\Hom_R(M'',T)\morphism\Hom_R(M,T)\morphism\Hom_R(M',T)\\
\morphism\Ext_R^1(M'',T)\morphism\Ext_R^1(M,T)\morphism\Ext_R^1(M',T)\morphism\Ext_R^2(M'',T)\morphism\ldots\;.
\end{multline*}
extending the open ends of \eqreff{eq:TensorSequence}, \eqreff{eq:HomSequence}, and \eqreff{eq:HomSequence2} respectively.

A highlight of this lecture will be \emph{Serre's characterization of regularity}.
\begin{thm*}
	For a  Noetherian local ring $R$ with maximal ideal $\mm$ and residue field $k$, the following are equivalent.
	\begin{alphanumerate}
		\item $\dim_k\mm/\mm^2=\dim R$ (i.e., $R$ is regular).
		\item There is some vanishing bound for $\Tor_*^R(-,-)$.
		\item \ldots and $\dim R$ is such a vanishing bound.
		\item There is some vanishing bound for $\Ext_R^*(-,-)$.
		\item \ldots and $\dim R$ is again such a vanishing bound.
	\end{alphanumerate}
\end{thm*}
From this, one can deduce the following
\begin{cor*}
	If $R$ is a regular Noetherian local ring and $\pp\in\Spec R$, then $R_\pp$ is regular as well.
\end{cor*}

We will also introduce the notion of \emph{Cohen--Macaulay rings} and prove that they are \emph{universally catenary} (which is quite a generalization of what we did in Algebra I, cf. \cite[Theorem~10]{alg1}).
\begin{thm*}
	If $R$ is a regular Noetherian local ring or, more generally, a Cohen--Macaulay ring, then it is \defemph{universally catenary}: If $A$ is an $R$-algebra of finite type and $X\subseteq Y\subseteq Z$ are irreducible closed subsets of $\Spec A$, then
	\begin{align*}
		\codim(X,Y)+\codim(Y,Z)=\codim(X,Z)\;.
	\end{align*}
\end{thm*}

\chapter{The $\Tor$ and $\Ext$ functors}
From now on, unless otherwise stated, our rings are commutative with $1$.
\section{Injective and projective modules}
\begin{prop}[Baer]\lbl{prop:InjectiveModules}
	For an $R$-module $N$, the following are equivalent.
	\begin{alphanumerate}
		\item The functor $\Hom_R(-,N)$ is exact.
		\item For any embedding $M'\monomorphism M$ of $R$-modules, $\Hom_R(M,N)\morphism\Hom_R(M',N)$ is surjective.
		\item Property \itememph{b} holds for $R=M$. In other words, if $I\subseteq R$ is any ideal, then any morphism $I\morphism N$ of $R$-modules extends to a morphism $R\morphism N$.
	\end{alphanumerate}
	\begin{rem*}
		\begin{alphanumerate}
			\item Since there is a bijection
			\begin{align*}
				\Hom_R(R,M)&\isomorphism M\\
				\left(r\mapsto r\cdot m\right)&\longmapsfrom m\\
				\left(R\morphism[\phi]M\right)&\longmapsto \phi(1)\;,
			\end{align*}
			Proposition~\reff{prop:InjectiveModules}\itememph{c} can be reformulated as that any morphism $I\morphism N$ for $I\subseteq R$ an ideal has the form $i\mapsto i\cdot m$ for some $m\in M$.
			\item Note that Proposition~\reff{prop:InjectiveModules}\itememph{c} is trivial when $I=0$.
			\item When $R=\IZ$, every ideal $I\subseteq \IZ$ has the form $n\IZ$ for some $n\in\IZ$ and a morphism $n\IZ\morphism[\phi]N$ is uniquely determined by $\phi(n)$. Thus, an extension $\hat\phi$ of $\phi$ to $\IZ$ exists iff there is an element $\nu\in N$ such that $n\cdot\nu=\phi(n)$ (in that case, put $\hat\phi(1)=\nu$). Hence, Proposition~\reff{prop:InjectiveModules}\itememph{c}  amounts to whether the abelian group $N$ is \emph{divisible}, that is, whether $N\morphism[\cdot n]N$ is surjective for all $n\in\IZ$.
		\end{alphanumerate}
	\end{rem*}
\end{prop}

\printbibliography

\end{document}          

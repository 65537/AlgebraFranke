\documentclass[a4paper,parskip=half,numbers=enddot, DIV=12]{scrreprt}
\usepackage[utf8]{inputenc}

\usepackage{../header}
\usepackage{../frankenumbering}
\usepackage{../shortcuts}

\usepackage{csquotes}
%\usepackage{tikz-cd}%I cannot draw diagrams without it - Felix. %well, I can - Ferdinand
\usepackage[backend=biber,style=numeric,sorting=none]{biblatex}
\setcounter{biburlnumpenalty}{7000}
\setcounter{biburllcpenalty}{7000}
\setcounter{biburlucpenalty}{8000}
\addbibresource{../literatur.bib}

% Title Page
\title{Homological Methods in Commutative Algebra}
\author{Ferdinand Wagner}
\date{Sommersemester 2018}

\displaywidowpenalty=8000
%\postdisplaypenalty=8000
\widowpenalty=8000
\clubpenalty=8000

\begin{document}
\pagenumbering{Alph}
\maketitle
\pagenumbering{roman}

\thispagestyle{plain}
This text consists of notes on the lecture Homological Methods in Commutative Algebra, taught at the University of Bonn by Professor Jens Franke in the summer term (Sommersemester) 2018. 

Please report bugs, typos etc. through the \emph{Issues} feature of github.

\tableofcontents

\addchap{Introduction}
\pagenumbering{arabic}

Professor Franke started the lecture giving an idea of what the $\Tor$ and $\Ext$ functors do. Let $R$ be a commutative ring with $1$. For an exact sequence of $R$-modules
\begin{align*}
	0\morphism M'\morphism M\morphism M''\morphism 0
\end{align*}
and $T$ another $R$-module, the sequence 
\begin{align}\lbl{eq:TensorSequence}
	M'\otimes_RT\morphism M\otimes_RT\morphism M''\otimes_RT\morphism 0
\end{align}
is exact but usually can't be extended by $0$ on the left end. The same is true for
\begin{align}\lbl{eq:HomSequence}
	0\morphism\Hom_R(T,M')\morphism\Hom_R(T,M)\morphism\Hom_R(T,M'')
\end{align}
and
\begin{align}\lbl{eq:HomSequence2}
	0\morphism\Hom_R(M'',T)\morphism\Hom_R(M,T)\morphism\Hom_R(M',T)\;,
\end{align}
but again, they can't be extended by $0$ on the right in general.
\begin{example*}
	Take $R=\IZ$ and consider the exact sequence $0\morphism\IZ\morphism[\cdot2]\IZ\morphism\IZ/2\IZ\morphism0$. 
	\begin{alphanumerate}
		\item Let $T=\IZ/2\IZ$ in \eqreff{eq:TensorSequence}. Then $\IZ\otimes_\IZ\IZ/2\IZ\simeq\IZ/2\IZ$ and $\IZ/2\IZ\morphism[\cdot 2]\IZ/2\IZ$ is the zero morphism, showing that injectivity on the left end fails in \eqreff{eq:TensorSequence}.
		\item Let $T=\IZ/2\IZ$ in \eqreff{eq:HomSequence}. We claim that surjectivity fails on the right end. Indeed, if it was surjective, then $\id_{\IZ/2\IZ}\in\Hom_\IZ(\IZ/2\IZ,\IZ/2\IZ)$ would have to have a lift
		\begin{diagram*}
			\node[ob] (Z2) at (0,0) {$\IZ/2\IZ$};			
			\node[ob] (Z22) at (1.75,0) {$\IZ/2\IZ$};
			\node[ob] (Z) at (1.75,1.5) {$\IZ$};
			\draw[transform canvas={yshift=1pt}] (Z2) -- (Z22);
			\draw[transform canvas={yshift=-1pt}] (Z2) -- (Z22);
			\draw[->>] (Z) -- (Z22);
			\draw[->, dashed] (Z2) -- (Z);
		\end{diagram*}
		which it hasn't as $\IZ$ is $2$-torsion free and thus every morphism $\IZ/2\IZ\morphism\IZ$ must be $0$.
		\item Let $T=\IZ$ in \eqreff{eq:HomSequence2}. We claim that that surjectivity fails on the right end, or more specifically, that $\id_{\IZ}\in\Hom_\IZ(\IZ,\IZ)$ has no preimage. Indeed, if $f\in\Hom_\IZ(\IZ,\IZ)$ is a preimage of $\id_\IZ$, i.e. the composition $\IZ\morphism[\cdot2]\IZ\morphism[f]\IZ$ equals $\id_\IZ$, then $f$ must be given by $f(n)=\frac{n}{2}$ on $2\IZ$, but this can't be extended to all of $\IZ$, contradiction!
	\end{alphanumerate}
\end{example*}
To handle this deficiency, one constructs \emph{derived functors} $\Tor$ and $\Ext$, which give rise to long exact sequences
\begin{multline*}
	\ldots\morphism\Tor_2^R(M'',T)\morphism \Tor_1^R(M',T)\morphism \Tor_1^R(M,T)\morphism \Tor_1^R(M'',T)\\
	\morphism M'\otimes_RT\morphism M\otimes_RT\morphism M''\otimes_RT\morphism 0\;,
\end{multline*}
as well as
\begin{multline*}
	0\morphism\Hom_R(T,M')\morphism\Hom_R(T,M)\morphism\Hom_R(T,M'')\\
	\morphism\Ext_R^1(T,M')\morphism\Ext_R^1(T,M)\morphism\Ext_R^1(T,M'')\morphism\Ext_R^2(T,M')\morphism\ldots
\end{multline*}
and
\begin{multline*}
0\morphism\Hom_R(M'',T)\morphism\Hom_R(M,T)\morphism\Hom_R(M',T)\\
\morphism\Ext_R^1(M'',T)\morphism\Ext_R^1(M,T)\morphism\Ext_R^1(M',T)\morphism\Ext_R^2(M'',T)\morphism\ldots\;.
\end{multline*}
extending the open ends of \eqreff{eq:TensorSequence}, \eqreff{eq:HomSequence}, and \eqreff{eq:HomSequence2} respectively.

A highlight of this lecture will be \emph{Serre's characterization of regularity}.
\begin{thm*}
	For a  Noetherian local ring $R$ with maximal ideal $\mm$ and residue field $k$, the following are equivalent.
	\begin{alphanumerate}
		\item $\dim_k\mm/\mm^2=\dim R$ (i.e., $R$ is regular).
		\item There is some vanishing bound for $\Tor_*^R(-,-)$.
		\item \ldots and $\dim R$ is such a vanishing bound.
		\item There is some vanishing bound for $\Ext_R^*(-,-)$.
		\item \ldots and $\dim R$ is again such a vanishing bound.
	\end{alphanumerate}
\end{thm*}
From this, one can deduce the following
\begin{cor*}
	If $R$ is a regular Noetherian local ring and $\pp\in\Spec R$, then $R_\pp$ is regular as well.
\end{cor*}

We will also introduce the notion of \emph{Cohen--Macaulay rings} and prove that they are \emph{universally catenary} (which is quite a generalization of what we did in Algebra I, cf. \cite[Theorem~10]{alg1}).
\begin{thm*}
	If $R$ is a regular Noetherian local ring or, more generally, a Cohen--Macaulay ring, then it is \defemph{universally catenary}: If $A$ is an $R$-algebra of finite type and $X\subseteq Y\subseteq Z$ are irreducible closed subsets of $\Spec A$, then
	\begin{align*}
		\codim(X,Y)+\codim(Y,Z)=\codim(X,Z)\;.
	\end{align*}
\end{thm*}

\chapter{$\Tor$ and $\Ext$ of $R$-modules}
From now on, unless otherwise stated, our rings are commutative with $1$.
\section{Injective and projective modules and properties of $\Ext_R^*$}
\begin{prop}[Baer's criterion]\lbl{prop:InjectiveModules}
	For an $R$-module $N$, the following are equivalent.
	\begin{alphanumerate}
		\item The functor $\Hom_R(-,N)$ is exact.
		\item For any embedding $M'\monomorphism M$ of $R$-modules, $\Hom_R(M,N)\morphism\Hom_R(M',N)$ is surjective.
		\item Property \itememph{b} holds for $R=M$. In other words, if $I\subseteq R$ is any ideal, then any morphism $I\morphism N$ of $R$-modules extends to a morphism $R\morphism N$.
	\end{alphanumerate}
\end{prop}
\begin{rem}
	\begin{alphanumerate}
		\item \lbl{rem:Injective}Since there is a bijection
		\begin{align*}
			\Hom_R(R,M)&\isomorphism M\\
			\left(r\mapsto r\cdot m\right)&\longmapsfrom m\\
			\left(R\morphism[\phi]M\right)&\longmapsto \phi(1)\;,
		\end{align*}
		Proposition~\reff{prop:InjectiveModules}\itememph{c} can be reformulated as that any morphism $I\morphism N$ for $I\subseteq R$ an ideal has the form $i\mapsto i\cdot m$ for some $m\in M$.
		\item Note that Proposition~\reff{prop:InjectiveModules}\itememph{c} is trivial when $I=0$.
		\item When $R=\IZ$, every ideal $I\subseteq \IZ$ has the form $n\IZ$ for some $n\in\IZ$ and a morphism $n\IZ\morphism[\phi]N$ is uniquely determined by $\phi(n)$. Thus, an extension $\hat\phi$ of $\phi$ to $\IZ$ exists iff there is an element $\nu\in N$ such that $n\cdot\nu=\phi(n)$ (in that case, put $\hat\phi(1)=\nu$). Hence, Proposition~\reff{prop:InjectiveModules}\itememph{c}  amounts to whether the abelian group $N$ is \emph{divisible}, that is, whether $N\morphism[\cdot n]N$ is surjective for all $n\in\IZ$ (also cf.\ Definition~\reff{def:Divisible}).
	\end{alphanumerate}
\end{rem}
\begin{defi}
	An $R$-module satisfying the equivalent conditions from Proposition~\reff{prop:InjectiveModules} is called \defemph{injective}.
\end{defi}
\begin{proof}[Proof of Proposition~\reff{prop:InjectiveModules}]
	The implication \itememph{b} $\Rightarrow$ \itememph{c} is trivial. Let's prove \itememph{c} $\Rightarrow$ \itememph{b}. Let $M\morphism[f]N$ be a morphism of $R$-modules and consider
	\begin{align*}
		\MM=\left\{(Q,\phi)\st M\subseteq Q\subseteq M'\text{ and }\snake{\phi}\in\Hom_R(Q,N)\text{ such that }\phi|_M=f\right\}\;.
	\end{align*}
	$\MM$ becomes a partially ordered set via $(Q_1,\phi_1)\preceq (Q_2,\phi_2)\Leftrightarrow Q_1\subseteq Q_2$ and $\phi_2|_{Q_1}=\phi_1$. Then it's easy to see that Zorn's lemma is applicable, hence $\MM$ has a $\preceq$-maximal element $(Q_*,\phi_*)$. If \itememph{c} is satisfied and $Q_*\subsetneq M'$, there is an $m\in M'\setminus L_*$. Let $I=\left\{r\in R\st rm\in Q_*\right\}$ and let $I\morphism[g]N$ be given by $g(r)=\phi_*(rm)$. By \itememph{c}, there is a morphism $R\morphism[\gamma]N$ extending $g$, i.e., a $\nu\in N$ such that $\phi_*(rm)=r\nu$ when $r\in I$ (using Remark~\reff{rem:Injective}\itememph{a}). Let $\snake{Q}=Q_*+Rm$ and $\snake{\phi}(m_*+rm)=\phi_*(m_*)+r\nu$ for $m_*\in Q_*$ and $r\in R$, then it's easy to see that $\snake{\phi}$ is well-defined and $(Q_*,\phi_*)\prec (\snake{Q},\snake{\phi})$, a contradiction.
	
	The equivalence \itememph{a} $\Leftrightarrow$ \itememph{b} is easy to see as for any short exact sequence $0\morphism X\morphism Y\morphism Z\morphism 0$, the sequence $0\morphism\Hom_R(Z,N)\morphism\Hom_R(Y,N)\morphism\Hom_R(X,N)$ is exact anyways and \itememph{b} implies exactness at the right end. 
\end{proof}
\begin{defi}\lbl{def:Divisible}
	If $R$ is a domain and $M$ an $R$-module, then $M$ is called \defemph{divisible} if $M\morphism[r\cdot]M$ is surjective for all $r\in R\setminus\{0\}$
\end{defi}
\begin{cor}
	\begin{alphanumerate}
		\item \lbl{cor:InjectivityAndDivisibility}When $R$ is a domain, the property from Proposition~\reff{prop:InjectiveModules}\itememph{c} for principal ideals $I$ is equivalent do divisibility of $N$.
		\item Any injective module $N$ is divisible in the following sense: If $r\in R$ is not a zero divisor, $N\morphism[r\cdot]N$ is surjective.
		\item In particular, if $N$ is injective and $S\subseteq R$ a multiplicative subset not containing zero divisors, then the morphism $N\morphism N_S$ to the localization of $N$ at $S$ is surjective.
	\end{alphanumerate}
\end{cor}
\begin{proof}
	Part \itememph{a} can be seen using the arguments from Remark~\reff{rem:Injective}\itememph{c}. For \itememph{b}, note that $R\morphism[r\cdot]R$ is injective when $r$ is no zero divisor, hence, for any $n\in N$, the morphism $\phi\in\Hom_R(R,N)$ given by $\phi(1)=n$ extends to $\hat{\phi}\in\Hom_R(R,N)$ such that $\phi=r\hat{\phi}$. Then $\hat{\phi}(1)$ is a preimage of $n$ under $N\morphism[r\cdot]N$. Part \itememph{c} follows from \itememph{b} and the universal property of localization.
\end{proof}
\begin{rem*}
	Note that $R=\IZ/p^2\IZ$, for $p\in\IZ$ a prime, is injective over itself, but $R\morphism[p\cdot ]R$ fails to be injective. Indeed, the only ideal of $R$ where Baer's criterion is in question is $(p)\subseteq R$. We need to show that any $R$-morphism $(p)\morphism R$ extends to an $R$-morphism $R\morphism R$. But any $(p)\morphism[\phi] R$ maps $p$ to the $p$-torsion part of $R$, i.e., to $(p)$ itself, hence is given by $\phi(p)=rp$ for some $r\in R$ and can be extended via $\hat{\phi}$ given by $\hat{\phi}(1)=r$. This shows that Corollary~\reff{cor:InjectivityAndDivisibility}\itememph{b} is somewhat sharp.
\end{rem*}
\begin{cor}
	A module over a principal ideal domain is injective iff it is divisible.
\end{cor}
\begin{proof}
	Follows from Corollary~\reff{cor:InjectivityAndDivisibility}\itememph{a}.
\end{proof}
\begin{rem*}
	The same holds for Dedekind domains, see Corollary~6 (which is not there yet).
\end{rem*}
\begin{cor}
	When $R$ is a principal ideal domain, then any quotient of an injective module is injective again. The category of $R$-modules has \defemph{sufficiently many injective objects} in the sense that for any object $X$ there is a monomophism $X\monomorphism I$ with $I$ injective. Thus, any $R$-module $X$ has an \defemph{injective resolution}, i.e., an exact sequence
	\begin{align*}
	0\morphism X\morphism I^0\morphism I^1\morphism I^2\morphism\ldots
	\end{align*}
	with injective objects $I^0,I^1,I^2,\ldots$. In fact, any $R$-module, for $R$ a principal ideal domain, has an injective resolution $0\morphism X\morphism I^0\morphism I^1\morphism 0$ of length $1$.
\end{cor}
\begin{proof}
	The first assertion follows as the quotient of divisible modules is divisible again. Note that $K/R$ is divisible, $K$ being the quotient field of $R$, hence it is injective. If $M$ is any $R$-module and $m\in M\setminus\{0\}$. We have to distinguish to cases.
	
	\emph{Case 1.} Suppose $\Ann_R(m)$ is non-zero, i.e., $\Ann_R(m)=(\alpha)$ for some $\alpha\in R\setminus \{0\}$ (remember we have a principal ideal domain). Then we have a morphism from $Rm\subseteq M$ to $K/R$ given by $rm\mapsto \frac{r}{\alpha}\bmod R$ (note that modding out $R$ is necessary for this to be well-defined -- we couldn't just have used $K$). By injectivity of $K/R$, there is an extension $M\morphism[\phi_m]K/R$, satisfying $\phi_m(m)\neq 0$. Let $I_m\subseteq K/R$ be the target of $\phi_m$.
	
	\emph{Case 2.} If $\Ann_R(m)=0$, we get a morphism from $Rm\subseteq M$ to $K$ instead, sending $rm\mapsto r$ (this time, using $K$ doesn't cause problems thanks to $\Ann_R(m)=0$). By injectivity of $K$, this extends to a morphism $M\morphism[\phi_m]K$ such that $\phi_m(m)\neq0$. Let $I_m=K$ be the target of $\phi_m$.
	
	Now put $I=\prod_{m\in M\setminus\{0\}}I_m$. Then $I$ is divisible (since every $I_m$ is), hence injective, and $M\morphism I$, $\mu\mapsto\left(\phi_m(\mu)\right)_{m\in M\setminus\{0\}}$ is a monomorphism. As a quotient of $I^0=I$, $I^1=\coker\left(M\morphism I^0\right)$ is injective as well, hence $0\morphism M\morphism I^0\morphism I^1\morphism 0$ is an injective resolution of length $1$.
\end{proof}
\begin{prop}[a.k.a. ``Satz 2'']\lbl{prop:RModHasEnoughInjectives}
	For any ring $R$, the category of $R$-modules has sufficiently many injective objects.
\end{prop}
\begin{proof}
	This will follow from Lemma~\reff{lem:AdjointStuff}\itememph{b} and \itememph{c} below.
\end{proof}
\begin{rem*}
	This holds in vast more generality, and in particular, Proposition~\reff{prop:RModHasEnoughInjectives} follows immediately from the following theorem, which, however, we are not going to prove in this lecture.
	\begin{thm*}[Grothendieck]
		Any AB5 category with a generator has sufficiently many injective objects.
	\end{thm*}
\end{rem*}
\begin{lem}\lbl{lem:AdjointStuff}
	Let $R$ be any ring.
	\begin{alphanumerate}
		\item The forgetful functor from $R\cat{-Mod}$ to the category of abelian groups has a right-adjoint functor, namely $\Hom_\IZ(R,-)$. That is, there is a bijection
		\begin{align}\lbl{eq:ForgetHomAdjunction}
			\Hom_\IZ(M,A)\isomorphism\Hom_R(M,\Hom_\IZ(R,A))\tag{$*$}
		\end{align}
		for any $R$-module $M$ and any abelian group $A$. Here, we equip $\Hom_\IZ(R,A)$ with an $R$-module structure via $(r\cdot\phi)(x)=\phi(rx)$ for $\phi\in\Hom_\IZ(R,A)$ and $r,x\in R$.
		\item For any injective abelian group $I$, $\Hom_\IZ(R,I)$ is an injective $R$-module.
		\item Let $M$ be any $R$-module and $I$ and abelian group and $M\monomorphism[\iota]I$ a monomorphism of abelian groups, then the $R$-morphism $M\morphism\Hom_\IZ(R,A)$ obtained by applying \eqreff{eq:ForgetHomAdjunction} is injective.
	\end{alphanumerate}
\end{lem}

\printbibliography

\end{document}          

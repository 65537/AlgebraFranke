\documentclass[a4paper,parskip=half,numbers=enddot, DIV=12]{scrreprt}
\usepackage[utf8]{inputenc}

\usepackage{../header}
\usepackage{../frankenumbering}
\usepackage{../shortcuts}

\usepackage{csquotes}
%\usepackage{tikz-cd}%I cannot draw diagrams without it - Felix. %well, I can - Ferdinand
\usepackage[backend=biber,style=numeric,sorting=none]{biblatex}
\setcounter{biburlnumpenalty}{7000}
\setcounter{biburllcpenalty}{7000}
\setcounter{biburlucpenalty}{8000}
\addbibresource{../literatur.bib}

% Title Page
\title{Homological Methods in Commutative Algebra}
\author{Ferdinand Wagner}
\date{Sommersemester 2018}

\displaywidowpenalty=8000
%\postdisplaypenalty=8000
\widowpenalty=8000
\clubpenalty=8000

\begin{document}
\pagenumbering{Alph}
\maketitle
\pagenumbering{roman}

\thispagestyle{plain}
This text consists of notes on the lecture Homological Methods in Commutative Algebra, taught at the University of Bonn by Professor Jens Franke in the summer term (Sommersemester) 2018. The dual lecture Cohomological Methods in Mmutative Algebra was given in the winter term (Wintersemester) 2017/18.

Please report bugs, typos etc. through the \emph{Issues} feature of github.

\tableofcontents

\addchap{Introduction}
\pagenumbering{arabic}

Professor Franke started the lecture giving an idea of what the $\Tor$ and $\Ext$ functors do. Let $R$ be a commutative ring with $1$. For an exact sequence of $R$-modules
\begin{align*}
	0\morphism M'\morphism M\morphism M''\morphism 0
\end{align*}
and $T$ another $R$-module, the sequence 
\begin{align}\lbl{eq:TensorSequence}
	M'\otimes_RT\morphism M\otimes_RT\morphism M''\otimes_RT\morphism 0
\end{align}
is exact but usually can't be extended by $0$ on the left end. The same is true for
\begin{align}\lbl{eq:HomSequence}
	0\morphism\Hom_R(T,M')\morphism\Hom_R(T,M)\morphism\Hom_R(T,M'')
\end{align}
and
\begin{align}\lbl{eq:HomSequence2}
	0\morphism\Hom_R(M'',T)\morphism\Hom_R(M,T)\morphism\Hom_R(M',T)\;,
\end{align}
but again, they can't be extended by $0$ on the right in general.
\begin{example*}
	Take $R=\IZ$ and consider the exact sequence $0\morphism\IZ\morphism[\cdot2]\IZ\morphism\IZ/2\IZ\morphism0$. 
	\begin{alphanumerate}
		\item Let $T=\IZ/2\IZ$ in \eqreff{eq:TensorSequence}. Then $\IZ\otimes_\IZ\IZ/2\IZ\simeq\IZ/2\IZ$ and $\IZ/2\IZ\morphism[\cdot 2]\IZ/2\IZ$ is the zero morphism, showing that injectivity on the left end fails in \eqreff{eq:TensorSequence}.
		\item Let $T=\IZ/2\IZ$ in \eqreff{eq:HomSequence}. We claim that surjectivity fails on the right end. Indeed, if it was surjective, then $\id_{\IZ/2\IZ}\in\Hom_\IZ(\IZ/2\IZ,\IZ/2\IZ)$ would have to have a lift
		\begin{diagram*}
			\node[ob] (Z2) at (0,0) {$\IZ/2\IZ$};			
			\node[ob] (Z22) at (1.75,0) {$\IZ/2\IZ$};
			\node[ob] (Z) at (1.75,1.5) {$\IZ$};
			\draw[transform canvas={yshift=1pt}] (Z2) -- (Z22);
			\draw[transform canvas={yshift=-1pt}] (Z2) -- (Z22);
			\draw[->>] (Z) -- (Z22);
			\draw[->, dashed] (Z2) -- (Z);
		\end{diagram*}
		which it hasn't as $\IZ$ is $2$-torsion free and thus every morphism $\IZ/2\IZ\morphism\IZ$ must be $0$.
		\item Let $T=\IZ$ in \eqreff{eq:HomSequence2}. We claim that that surjectivity fails on the right end, or more specifically, that $\id_{\IZ}\in\Hom_\IZ(\IZ,\IZ)$ has no preimage. Indeed, if $f\in\Hom_\IZ(\IZ,\IZ)$ is a preimage of $\id_\IZ$, i.e. the composition $\IZ\morphism[\cdot2]\IZ\morphism[f]\IZ$ equals $\id_\IZ$, then $f$ must be given by $f(n)=\frac{n}{2}$ on $2\IZ$, but this can't be extended to all of $\IZ$, contradiction!
	\end{alphanumerate}
\end{example*}
To handle this deficiency, one constructs \emph{derived functors} $\Tor$ and $\Ext$, which give rise to long exact sequences
\begin{multline*}
	\ldots\morphism\Tor_2^R(M'',T)\morphism \Tor_1^R(M',T)\morphism \Tor_1^R(M,T)\morphism \Tor_1^R(M'',T)\\
	\morphism M'\otimes_RT\morphism M\otimes_RT\morphism M''\otimes_RT\morphism 0\;,
\end{multline*}
as well as
\begin{multline*}
	0\morphism\Hom_R(T,M')\morphism\Hom_R(T,M)\morphism\Hom_R(T,M'')\\
	\morphism\Ext_R^1(T,M')\morphism\Ext_R^1(T,M)\morphism\Ext_R^1(T,M'')\morphism\Ext_R^2(T,M')\morphism\ldots
\end{multline*}
and
\begin{multline*}
0\morphism\Hom_R(M'',T)\morphism\Hom_R(M,T)\morphism\Hom_R(M',T)\\
\morphism\Ext_R^1(M'',T)\morphism\Ext_R^1(M,T)\morphism\Ext_R^1(M',T)\morphism\Ext_R^2(M'',T)\morphism\ldots\;.
\end{multline*}
extending the open ends of \eqreff{eq:TensorSequence}, \eqreff{eq:HomSequence}, and \eqreff{eq:HomSequence2} respectively.

A highlight of this lecture will be \emph{Serre's characterization of regularity}.
\begin{thm*}
	For a  Noetherian local ring $R$ with maximal ideal $\mm$ and residue field $k$, the following are equivalent.
	\begin{alphanumerate}
		\item $\dim_k\mm/\mm^2=\dim R$ (i.e., $R$ is regular).
		\item There is some vanishing bound for $\Tor_*^R(-,-)$.
		\item \ldots and $\dim R$ is such a vanishing bound.
		\item There is some vanishing bound for $\Ext_R^*(-,-)$.
		\item \ldots and $\dim R$ is again such a vanishing bound.
	\end{alphanumerate}
\end{thm*}
From this, one can deduce the following
\begin{cor*}
	If $R$ is a regular Noetherian local ring and $\pp\in\Spec R$, then $R_\pp$ is regular as well.
\end{cor*}

We will also introduce the notion of \emph{Cohen--Macaulay rings} and prove that they are \emph{universally catenary} (which is quite a generalization of what we did in Algebra I, cf. \cite[Theorem~10]{alg1}).
\begin{thm*}
	If $R$ is a regular Noetherian local ring or, more generally, a Cohen--Macaulay ring, then it is \defemph{universally catenary}: If $A$ is an $R$-algebra of finite type and $X\subseteq Y\subseteq Z$ are irreducible closed subsets of $\Spec A$, then
	\begin{align*}
		\codim(X,Y)+\codim(Y,Z)=\codim(X,Z)\;.
	\end{align*}
\end{thm*}

\chapter{$\Tor$ and $\Ext$ of $R$-modules}
From now on, unless otherwise stated, our rings are commutative with $1$.
\section{Injective and projective modules and properties of $\Ext_R^*$}
\begin{prop}[Baer's criterion]\lbl{prop:InjectiveModules}
	For an $R$-module $N$, the following are equivalent.
	\begin{alphanumerate}
		\item The functor $\Hom_R(-,N)$ is exact.
		\item For any embedding $M'\monomorphism M$ of $R$-modules, $\Hom_R(M,N)\morphism\Hom_R(M',N)$ is surjective.
		\item Property \itememph{b} holds for $R=M$. In other words, if $I\subseteq R$ is any ideal, then any morphism $I\morphism N$ of $R$-modules extends to a morphism $R\morphism N$.
	\end{alphanumerate}
\end{prop}
\begin{rem}
	\begin{alphanumerate}
		\item \lbl{rem:Injective}Since there is a bijection
		\begin{align*}
			\Hom_R(R,M)&\isomorphism M\\
			\left(r\mapsto r\cdot m\right)&\longmapsfrom m\\
			\left(R\morphism[\phi]M\right)&\longmapsto \phi(1)\;,
		\end{align*}
		Proposition~\reff{prop:InjectiveModules}\itememph{c} can be reformulated as that any morphism $I\morphism N$ for $I\subseteq R$ an ideal has the form $i\mapsto i\cdot m$ for some $m\in M$.
		\item Note that Proposition~\reff{prop:InjectiveModules}\itememph{c} is trivial when $I=0$.
		\item When $R=\IZ$, every ideal $I\subseteq \IZ$ has the form $n\IZ$ for some $n\in\IZ$ and a morphism $n\IZ\morphism[\phi]N$ is uniquely determined by $\phi(n)$. Thus, an extension $\hat\phi$ of $\phi$ to $\IZ$ exists iff there is an element $\nu\in N$ such that $n\cdot\nu=\phi(n)$ (in that case, put $\hat\phi(1)=\nu$). Hence, Proposition~\reff{prop:InjectiveModules}\itememph{c}  amounts to whether the abelian group $N$ is \emph{divisible}, that is, whether $N\morphism[\cdot n]N$ is surjective for all $n\in\IZ$ (also cf.\ Definition~\reff{def:Divisible}).
	\end{alphanumerate}
\end{rem}
\begin{defi}
	\begin{alphanumerate}
		\item An $R$-module is called \defemph{injective} if it satisfies the equivalent conditions from Proposition~\reff{prop:InjectiveModules}.
		\item In an arbitrary category $\Aa$, an object $I$ is called \defemph{injective} if for every monomorphism $X\monomorphism Y$, the induced map $\Hom_\Aa(Y,I)\morphism\Hom_\Aa(X,I)$ is surjective, that is, for every morphism $X\morphism[\phi]I$ there is a (usually non-unique) lift
		\begin{diagram*}
			\node[ob] (X) at (0,0) {$X$};			
			\node[ob] (Y) at (1.75,0) {$Y$};
			\node[ob] (I) at (0,1.5) {$I$};
			\scriptsize
			\draw[right hook->] (X) -- (Y);
			\draw[->] (X) -- (I) node[pos=0.5, left] {$\phi$};
			\draw[->, dashed] (Y) -- (I) node[pos=0.5, above right] {$\exists\ \hat{\phi}$};
		\end{diagram*}
	\end{alphanumerate}
\end{defi}
\begin{proof}[Proof of Proposition~\reff{prop:InjectiveModules}]
	The implication \itememph{b} $\Rightarrow$ \itememph{c} is trivial. Let's prove \itememph{c} $\Rightarrow$ \itememph{b}. Let $M\morphism[f]N$ be a morphism of $R$-modules and consider
	\begin{align*}
		\MM=\left\{(Q,\phi)\st M\subseteq Q\subseteq M'\text{ and }\snake{\phi}\in\Hom_R(Q,N)\text{ such that }\phi|_M=f\right\}\;.
	\end{align*}
	$\MM$ becomes a partially ordered set via $(Q_1,\phi_1)\preceq (Q_2,\phi_2)\Leftrightarrow Q_1\subseteq Q_2$ and $\phi_2|_{Q_1}=\phi_1$. Then it's easy to see that Zorn's lemma is applicable, hence $\MM$ has a $\preceq$-maximal element $(Q_*,\phi_*)$. If \itememph{c} is satisfied and $Q_*\subsetneq M'$, there is an $m\in M'\setminus L_*$. Let $I=\left\{r\in R\st rm\in Q_*\right\}$ and let $I\morphism[g]N$ be given by $g(r)=\phi_*(rm)$. By \itememph{c}, there is a morphism $R\morphism[\gamma]N$ extending $g$, i.e., a $\nu\in N$ such that $\phi_*(rm)=r\nu$ when $r\in I$ (using Remark~\reff{rem:Injective}\itememph{a}). Let $\snake{Q}=Q_*+Rm$ and $\snake{\phi}(m_*+rm)=\phi_*(m_*)+r\nu$ for $m_*\in Q_*$ and $r\in R$, then it's easy to see that $\snake{\phi}$ is well-defined and $(Q_*,\phi_*)\prec (\snake{Q},\snake{\phi})$, a contradiction.
	
	The equivalence \itememph{a} $\Leftrightarrow$ \itememph{b} is easy to see as for any short exact sequence $0\morphism X\morphism Y\morphism Z\morphism 0$, the sequence $0\morphism\Hom_R(Z,N)\morphism\Hom_R(Y,N)\morphism\Hom_R(X,N)$ is exact anyways and \itememph{b} implies exactness at the right end. 
\end{proof}
\begin{defi}\lbl{def:Divisible}
	If $R$ is a domain and $M$ an $R$-module, then $M$ is called \defemph{divisible} if $M\morphism[r\cdot]M$ is surjective for all $r\in R\setminus\{0\}$
\end{defi}
\begin{cor}
	\begin{alphanumerate}
		\item \lbl{cor:InjectivityAndDivisibility}When $R$ is a domain, the property from Proposition~\reff{prop:InjectiveModules}\itememph{c} for principal ideals $I$ is equivalent do divisibility of $N$.
		\item Any injective module $N$ is divisible in the following sense: If $r\in R$ is not a zero divisor, $N\morphism[r\cdot]N$ is surjective.
		\item In particular, if $N$ is injective and $S\subseteq R$ a multiplicative subset not containing zero divisors, then the morphism $N\morphism N_S$ to the localization of $N$ at $S$ is surjective.
	\end{alphanumerate}
\end{cor}
\begin{proof}
	Part \itememph{a} can be seen using the arguments from Remark~\reff{rem:Injective}\itememph{c}. For \itememph{b}, note that $R\morphism[r\cdot]R$ is injective when $r$ is no zero divisor, hence, for any $n\in N$, the morphism $\phi\in\Hom_R(R,N)$ given by $\phi(1)=n$ extends to $\hat{\phi}\in\Hom_R(R,N)$ such that $\phi=r\hat{\phi}$. Then $\hat{\phi}(1)$ is a preimage of $n$ under $N\morphism[r\cdot]N$. Part \itememph{c} follows from \itememph{b} and the universal property of localization.
\end{proof}
\begin{rem*}
	Note that $R=\IZ/p^2\IZ$, for $p\in\IZ$ a prime, is injective over itself, but $R\morphism[p\cdot ]R$ fails to be injective. Indeed, the only ideal of $R$ where Baer's criterion is in question is $(p)\subseteq R$. We need to show that any $R$-morphism $(p)\morphism R$ extends to an $R$-morphism $R\morphism R$. But any $(p)\morphism[\phi] R$ maps $p$ to the $p$-torsion part of $R$, i.e., to $(p)$ itself, hence is given by $\phi(p)=rp$ for some $r\in R$ and can be extended via $\hat{\phi}$ given by $\hat{\phi}(1)=r$. This shows that Corollary~\reff{cor:InjectivityAndDivisibility}\itememph{b} is somewhat sharp.
\end{rem*}
\begin{cor}
	A module over a principal ideal domain is injective iff it is divisible.
\end{cor}
\begin{proof}
	Follows from Corollary~\reff{cor:InjectivityAndDivisibility}\itememph{a}.
\end{proof}
\begin{rem*}
	The same holds for Dedekind domains, see Corollary~6 (which is not there yet).
\end{rem*}
\begin{cor}
	When $R$ is a principal ideal domain, then any quotient of an injective module is injective again. The category of $R$-modules has \defemph{sufficiently many injective objects} in the sense that for any object $X$ there is a monomophism $X\monomorphism I$ with $I$ injective. Thus, any $R$-module $X$ has an \defemph{injective resolution}, i.e., an exact sequence
	\begin{align*}
	0\morphism X\morphism I^0\morphism I^1\morphism I^2\morphism\ldots
	\end{align*}
	with injective objects $I^0,I^1,I^2,\ldots$. In fact, any $R$-module, for $R$ a principal ideal domain, has an injective resolution $0\morphism X\morphism I^0\morphism I^1\morphism 0$ of length $1$.
\end{cor}
\begin{proof}
	The first assertion follows as the quotient of divisible modules is divisible again. Note that $K/R$ is divisible, $K$ being the quotient field of $R$, hence it is injective. If $M$ is any $R$-module and $m\in M\setminus\{0\}$. We have to distinguish to cases.
	
	\emph{Case 1.} Suppose $\Ann_R(m)$ is non-zero, i.e., $\Ann_R(m)=(\alpha)$ for some $\alpha\in R\setminus \{0\}$ (remember we have a principal ideal domain). Then we have a morphism from $Rm\subseteq M$ to $K/R$ given by $rm\mapsto \frac{r}{\alpha}\bmod R$ (note that modding out $R$ is necessary for this to be well-defined -- we couldn't just have used $K$). By injectivity of $K/R$, there is an extension $M\morphism[\phi_m]K/R$, satisfying $\phi_m(m)\neq 0$. Let $I_m\subseteq K/R$ be the target of $\phi_m$.
	
	\emph{Case 2.} If $\Ann_R(m)=0$, we get a morphism from $Rm\subseteq M$ to $K$ instead, sending $rm\mapsto r$ (this time, using $K$ doesn't cause problems thanks to $\Ann_R(m)=0$). By injectivity of $K$, this extends to a morphism $M\morphism[\phi_m]K$ such that $\phi_m(m)\neq0$. Let $I_m=K$ be the target of $\phi_m$.
	
	Now put $I=\prod_{m\in M\setminus\{0\}}I_m$. Then $I$ is divisible (since every $I_m$ is), hence injective, and $M\morphism I$, $\mu\mapsto\left(\phi_m(\mu)\right)_{m\in M\setminus\{0\}}$ is a monomorphism. As a quotient of $I^0=I$, $I^1=\coker\left(M\morphism I^0\right)$ is injective as well, hence $0\morphism M\morphism I^0\morphism I^1\morphism 0$ is an injective resolution of length $1$.
\end{proof}
\begin{prop}[a.k.a. ``Satz 2'']\lbl{prop:RModHasEnoughInjectives}
	For any ring $R$, the category of $R$-modules has sufficiently many injective objects.
\end{prop}
\begin{proof}
	This will follow from Lemma~\reff{lem:AdjointStuff}\itememph{b} and \itememph{c} below.
\end{proof}
\begin{rem*}
	This holds in vast more generality, and in particular, Proposition~\reff{prop:RModHasEnoughInjectives} follows immediately from the following theorem, which, however, we are not going to prove in this lecture.
	\begin{thm*}[Grothendieck]
		Any AB5 category with a generator has sufficiently many injective objects.
	\end{thm*}
\end{rem*}
\begin{lem}\lbl{lem:AdjointStuff}
	Let $R$ be any ring.
	\begin{alphanumerate}
		\item The forgetful functor from $R\cat{-Mod}$ to the category $\IZ\cat{-Mod}$ of abelian groups has a right-adjoint functor, namely $\Hom_\IZ(R,-)$. That is, there is a bijection
		\begin{align}\lbl{eq:ForgetHomAdjunction}
			\Hom_\IZ(M,A)\isomorphism\Hom_R(M,\Hom_\IZ(R,A))\tag{$*$}
		\end{align}
		for any $R$-module $M$ and any abelian group $A$. Here, we equip $\Hom_\IZ(R,A)$ with an $R$-module structure via $(r\cdot\phi)(x)=\phi(xr)$ for $\phi\in\Hom_\IZ(R,A)$ and $r,x\in R$.
		\item For any injective abelian group $I$, $\Hom_\IZ(R,I)$ is an injective $R$-module.
		\item Let $M$ be any $R$-module and $I$ and abelian group and $M\monomorphism[\phi]I$ a monomorphism of abelian groups, then the $R$-morphism $M\morphism\Hom_\IZ(R,I)$ obtained by applying \eqreff{eq:ForgetHomAdjunction} is injective.
	\end{alphanumerate}
\end{lem}
\begin{proof}
	Part \itememph{a}. The proof given in the lecture was rather computational, so I decided to include a more elegant one. It is easy to see that $\Hom_\IZ(R,-)$ is indeed a functor $\IZ\cat{-Mod}\morphism R\cat{-Mod}$. From the well-known tensor-hom adjunction we obtain a canonical bijection
	\begin{align*}
		\Hom_\IZ(M\otimes _RR,A)\isomorphism\Hom_R(M,\Hom_\IZ(R,A))\;.
	\end{align*}
	But $M$ is an $R$-module and so $M\otimes_RR\simeq M$ canonically, proving \eqreff{eq:ForgetHomAdjunction}. 
	%The bijection-in-question \eqreff{eq:ForgetHomAdjunction} sends $\phi\in\Hom_\IZ(M,A)$ to $\psi\in\Hom_R(M,\Hom_\IZ(R,A))$ given by $(\psi(m))(r)=\phi(rm)$ and 
	
	Part \itememph{b}. Since the forgetful functor $R\cat{-Mod}\morphism\IZ\cat{-Mod}$ clearly preserves injectivity of morphisms (i.e., monomorphisms), this comes down to the following more general fact about adjoint pairs of functors.
	\begin{fact}\lbl{fact:AdjointInjProj}
		Let $\Aa\doublelrmorphism[L][R]\Bb$ be an adjoint pairs of functors. Suppose that $L$ preserves monomorphisms. Then $R$ preserves injective objects.
	\end{fact}
	\begin{proof}[Proof of Fact~\reff{fact:AdjointInjProj}]
		Let $I\in\Ob(\Bb)$ be injective and $X\monomorphism Y$ be a monomorphism in $\Aa$. By assumption, $LX\monomorphism  LY$ is a monomorphism in $\Bb$. In the diagram 
			\begin{diagram*}
				\node[ob] (Y) at (0,1.5) {$\Hom_\Aa(Y,RI)$};
				\node[ob] (LY) at (0,0) {$\Hom_\Bb(LY,I)$};
				\node[ob] (X) at (3.5,1.5) {$\Hom_\Aa(X,RI)$};
				\node[ob] (LX) at (3.5,0) {$\Hom_\Bb(LX,I)$};
				\scriptsize
				\draw[->] (Y) -- (LY) node[pos=0.5, sloped, above=-0.25ex] {$\sim$};
				\draw[->] (X) -- (LX) node[pos=0.5, sloped, above=-0.25ex] {$\sim$};
				\draw[->] (Y) -- (X);
				\draw[->] (LY) -- (LX);
			\end{diagram*}
		the lower horizontal arrow is surjective by injectivity of $I$, hence so is the upper horizontal arrow.
	\end{proof}
	Back to the proof of Lemma~\reff{lem:AdjointStuff} and let's prove \itememph{c}. Let $M\monomorphism[\phi]I$ be a monomorphism of abelian groups. The corresponding morphism $M\morphism[\psi]\Hom_\IZ(R,I)$ sends $m\in M$ to $\psi(m)\colon R\morphism I$ given by $\psi(m)(r)=\phi(rm)$. If $\psi(m)$ is the zero morphism for some $m\in M$, then $0=\psi(m)(1)=\phi(m)$, proving $m=0$ by injectivity of $\phi$. Then $\psi$ is also injective.
\end{proof}

\appendix
\chapter{Appendix -- category theory corner}
\setcounter{thm}{0}
\renewcommand*{\thethm}{\Alph{thm}}

\section{Derived functors and $\Ext_R^*$}
\begin{defi}
	Let $\Aa$ and $\Bb$ be abelian categories (cf. \cite[Definition~A.1.4]{alggeo2}). A \defemph{homological $\partial$-functor} $F_*\colon\Aa\morphism\Bb$ is a sequence $(F_n)_{n\geq 0}$  of additive functors $\Aa\morphism[F_i]\Bb$ together with a natural transformation $\partial=\partial_F\colon F_{i+1}(A'')\morphism F_i(A')$ on the category of short exact sequences on $\Aa$, such that the sequence
	\begin{multline*}
		\ldots\morphism F_{i+1}(A'')\morphism[\partial]F_i(A')\morphism F_i(A)\morphism F_i(A'')\morphism[\partial]\ldots\\
		\ldots\morphism F_{1}(A'')\morphism[\partial]F_0(A')\morphism F_0(A)\morphism F_0(A'')\morphism 0\;.
	\end{multline*}
	is exact whenever $0\morphism A'\morphism A\morphism A''\morphism 0$ is a short exact sequence in $\Aa$.
	
	A \defemph{morphism} $F_*\morphism[\phi]G_*$ of homological $\partial$-functors is a sequence $(\phi_n)_{n\geq0}$ of natural transformations $F_i\morphism[\phi_i]G_i$ such that for any short exact sequence $0\morphism A'\morphism A\morphism A''\morphism 0$ in $\Aa$ the diagram 
	\begin{diagram*}
		\node[ob] (F1A) at (0,1.5) {$F_{i+1}(A'')$};
		\node[ob] (G1A) at (0,0) {$G_{i+1}(A'')$};
		\node[ob] (FA) at (2.5,1.5) {$F_i(A')$};
		\node[ob] (GA) at (2.5,0) {$G_i(A')$};
		\scriptsize
		\draw[->] (F1A) -- (G1A) node[pos=0.5, left] {$\phi_{i+1}$};
		\draw[->] (FA) -- (GA) node[pos=0.5, right] {$\phi_{i}$};
		\draw[->] (F1A) -- (FA) node[pos=0.5, above] {$\partial_F$};
		\draw[->] (G1A) -- (GA) node[pos=0.5, above] {$\partial_G$};
	\end{diagram*}
	commutes.
	
	Similarly, a \defemph{cohomological $\partial$-functor} $F^*\colon\Aa\morphism\Bb$ is a sequence $(F^n)_{n\geq 0}$ of additive functors $\Aa\morphism[F^i]\Bb$ together with connecting morphism $\partial=\partial_F\colon F^i(A'')\morphism F^{i+1}(A')$ such that for a short exact sequence $0\morphism A'\morphism A\morphism A''\morphism 0$, the sequence 
	\begin{multline*}
		0\morphism F^0(A')\morphism F^0(A)\morphism F^0(A'')\morphism[\partial] F^1(A')\morphism\ldots\\
		\ldots\morphism[\partial]F^i(A')\morphism F^i(A)\morphism F^i(A'')\morphism[\partial]F^{i+1}(A')\morphism\ldots
	\end{multline*}
	 is required to be exact. And the notion of a \defemph{morphism} $F_*\morphism[\phi]G_*$ of cohomological $\partial$-functors is defined in the obvious way.
	
	Let $\Aa\morphism[F]\Bb$ be a \emph{right-exact} functor, i.e., for any short exact sequence $0\morphism A'\morphism A\morphism A''\morphism 0$, the sequence $FA'\morphism FA\morphism FA''\morphism 0$ is exact. A \defemph{left-derived functor} of $F$ is a homological functor $L_*F$ from $\Aa$ to $\Bb$ with a natural isomorphism $L_0F\simeq F$ such that for any homological functor $\Phi_*\colon \Aa\morphism\Bb$, any natural transformation $\Phi_0\morphism L_0F$ extends in a unique way to a morphism $\Phi_*\morphism L_*F$ of homological functors.
	
	Similar, a functor $\Aa\morphism[F]\Bb$ is \emph{right-exact} functor if $0\morphism FA'\morphism FA\morphism FA''$ is exact for any short exact sequence $0\morphism A'\morphism A\morphism A''\morphism 0$. A \defemph{right-derived functor} of $F$ is a homological functor $R^*F$ from $\Aa$ to $\Bb$ with a natural isomorphism $R^0F\simeq F$ such that for any homological functor $\Psi^*\colon \Aa\morphism\Bb$, any natural transformation $R^0F\morphism \Psi^0$ extends in a unique way to a morphism $R^*F\morphism \Psi^*$ of homological functors.
\end{defi}
\begin{rem}
	\begin{alphanumerate}
		\item It follows (by the usual Yoneda argument) that derived functors are unique up to unique isomorphism of (co)homological functors if they exist.
		\item If $F$ is left-exact in the above sense, it preserves monomorphisms and it can be shown that $0\morphism FX'\morphism FX\morphism FX''$ is exact even when only $0\morphism X'\morphism X\morphism X''$ is exact (a nasty technical proof which won't appear here). Similar for right-exact functors.
		\item A generalized definition drops the exactness assumptions and requires $F\morphism L_0F$ with the universal property that any diagram
		\begin{diagram*}
			\node[ob] (F) at (0,1.25) {$F$};
			\node[ob] (L) at (2.5,1.25) {$L_0F$};
			\node[ob] (Phi) at (1.25,0) {$\Phi_0$};
			\draw[->] (F) -- (L);
			\draw[->] (F) -- (Phi);
			\draw[->] (Phi) -- (L);
		\end{diagram*}
		can be uniquely extended to a morphism $\Phi_*\morphism L_*F$ of homological functors. Similar for right-derived functors.
	\end{alphanumerate}
\end{rem}
\begin{example}
	If $F$ is an exact functor, then left- and right-derived functors of $F$ are given by $L_0F=R^0F=F$ and $L_iF=R^iF=0$ for $i\geq 1$.
\end{example}
\begin{defi}
	An object $I$ in an abelian category $\Aa$ is injective iff the following equivalent conditions hold.
	\begin{alphanumerate}
		\item When $X\monomorphism[\xi] Y$ is a monomorphism, then any morphism $X\morphism[\iota] I$ extends to a morphism $Y\morphism I$.
		\item Any short exact sequence $0\morphism I\morphism X\morphism X''\morphism 0$ splits.
	\end{alphanumerate}
\end{defi}
\begin{proof}
	To see \itememph{a} $\Rightarrow$ \itememph{b}, extend $\id_I$ to $X\morphism[\pi]I$, then $\pi$ gives a split of the exact sequence (the argument used in the case of $R$-modules still works in arbitrary abelian categories).
	
	For \itememph{b} $\Rightarrow$ \itememph{a} consider $C=\coker\Big(X\xrightarrow{\iota\times\xi}I\oplus Y\Big)$ and let $I\oplus Y\morphism[p]C$ be the associated morphism. Let $i=\id_I\times 0$ and $j=0\times\id_Y$ be the canonical inclusions $I\morphism I\oplus Y$ and $Y\morphism I\oplus Y$. We claim that the composition 
	\begin{align*}
		I\morphism[i] I\oplus Y\morphism[p] C
	\end{align*}
	is a monomorphism. First note that $X\xrightarrow{\iota\times\xi}I\oplus Y$ is a monomorphism (since its composition with the projection to $Y$ equals $\xi$, which was supposed to be monic), hence it's the kernel of its own cokernel as we are working in an abelian category and thus every monomorphism is an \emph{effective monomorphism}, cf. \cite[Definition~A.1.3\itememph{d} and Definition~A.1.4]{alggeo2}. That is, $X=\ker(p)$. Suppose now that $T\morphism[\tau]I$ is a morphism satisfying $pi\tau=0$, then $i\tau$ factors over $X=\ker(p)$. We thus have a diagram
	\begin{diagram}[baseline=0.75cm-0.5ex][\lbl{diag:kernelStuff}]
		\node[ob] (t) at (0,1.5) {$T$};
		\node[ob] (x) at (0,0) {$X$};
		\node[ob] (i) at (2.5,1.5) {$I$};
		\node[ob] (iy) at (2.5,0) {$I\oplus Y$};
		\scriptsize
		\draw[->, dashed] (t) -- (x) node[pos=0.5, left] {$!\exists\ \vartheta$};
		\draw[->] (t) -- (i) node[pos=0.5, above] {$\tau$};
		\draw[->] (i)-- (iy) node[pos=0.5, right] {$i$};
		\draw[->] (x) -- (iy) node[pos=0.5, above] {$\iota\times\xi$};
		\tag{$*$}
	\end{diagram}
	Postcomposing with the canonical projection $I\oplus Y\epimorphism[\pi] Y$ we see that $\pi i\tau=0\circ\tau=0$, hence also
	$\xi\vartheta=0$ as \eqreff{diag:kernelStuff} commutes and $\pi\circ(\iota\times\xi)=\xi$. But $\xi$ is a monomorphism, hence $\vartheta=0$. By \eqreff{diag:kernelStuff}, this implies $\tau=0$ as $i$ is a monomorphism. This shows that $\alpha$ is indeed a monomorphism.
	
	We thus obtain a short exact sequence
	\begin{align*}
		0\morphism I\morphism C\morphism\coker(\alpha)\morphism 0
	\end{align*}
	which splits due to \itememph{b}, i.e., $C\simeq I\oplus \coker(\alpha)$. Let $C\epimorphism[q] I$ be the associated projection. Consider the composition
	\begin{align*}
		Y\morphism[j] I\oplus Y\morphism[p] C\epimorphism[q] I\;.
	\end{align*}
	We claim that $\upsilon=-qpj$ is a morphism $Y\morphism[\upsilon] I$ extending $X\morphism[\iota]I$. We have $qp\circ(\iota\times\xi)=q\circ0=0$ since $C$ is precisely the cokernel of $\iota\times\xi$. Also $qp\circ (\iota\times0)=\iota$ as $qpi=\id_I$ by construction of $q$. Then 
	\begin{align*}
		\upsilon\xi=-qpj\xi=-qp\circ(0\times\xi)=qp\circ\big((\iota\times 0)-(\iota\times\xi)\big)=\iota-0=\iota\;,
	\end{align*}
	hence $\upsilon$ has indeed the required property.
\end{proof}
\begin{thm}
	Let $\Aa$ be an abelian category with sufficiently many injective objects.
	\begin{alphanumerate}
		\item Any left-exact functor $\Aa\morphism\Bb$ from $\Aa$ to any abelian category $\Bb$ has a right-derived functor.
		\item Let $\Phi^*\colon \Aa\morphism\Bb$ be a cohomological functor, then $\Phi^*$ is a right-derived functor of $\Phi^0$ iff $\Phi^iI=0$ for any injective object $I\in\Ob(\Aa)$ and all $i\geq 1$.
		\item Let $\Ff\colon 0\morphism \Phi'\morphism\Phi\morphism\Phi''\morphism0$ be a sequence of left-exact functors $\Aa\morphism\Bb$ and functor morphisms between them such that $0\morphism\Phi'I\morphism\Phi I\morphism\Phi'' I\morphism 0$ is exact when $I$ is an injective object of $\Aa$. Then there is a unique sequence of natural transformations $R^i\Phi''\morphism[d_\Ff]R^{i+1}\Phi'$ such that
		\begin{multline*}
			0\morphism\Phi'X\morphism\Phi X\morphism\Phi''X\morphism[d_\Ff]R^1\Phi'X\morphism\ldots\\
			\ldots\morphism R^{i-1}\Phi''X\morphism[d_\Ff]R^i\Phi'X\morphism R^i\Phi X\morphism R^i\Phi''X\morphism[d_\Ff]R^{i+1}\Phi'X\morphism\ldots
		\end{multline*}
		is exact for arbitrary $X\in\Ob(\Aa)$ and such that the diagram
		\begin{diagram*}
			\node[ob] (F1A) at (0,1.5) {$R^i\Phi''X''$};
			\node[ob] (G1A) at (0,0) {$R^{i+1}\Phi''X'$};
			\node[ob] (FA) at (3,1.5) {$R^{i+1}\Phi'X''$};
			\node[ob] (GA) at (3,0) {$R^{i+2}\Phi'X'$};
			\scriptsize
			\draw[->] (F1A) -- (G1A) node[pos=0.5, left] {$\partial_{R^*\Phi''}$};
			\draw[->] (FA) -- (GA) node[pos=0.5, right] {$-\partial_{R^*\Phi'}$};
			\draw[->] (F1A) -- (FA) node[pos=0.5, above] {$d_\Ff$};
			\draw[->] (G1A) -- (GA) node[pos=0.5, above] {$d_\Ff$};
		\end{diagram*}
		commutes. Note the minus sign on the right vertical arrow!
	\end{alphanumerate}
\end{thm}
\printbibliography

\end{document}          

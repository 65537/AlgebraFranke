\documentclass[DIV=14,parskip=full,pointednumbers]{scrartcl}
\usepackage[utf8]{inputenc}
\setkomafont{sectioning}{\rmfamily\bfseries}

\usepackage{amsmath,amsfonts,amssymb,mathtools}
\usepackage{amsthm}
\usepackage{hyperref}
\usepackage{xstring}
\usepackage{bm}

\usepackage{old-arrows}
\usepackage{pgf,tikz}
\usetikzlibrary{arrows}
\usepackage[shortlabels]{enumitem}

\usepackage[english]{babel}
\usetikzlibrary{calc}
\usepackage{subcaption}

\renewcommand{\phi}{\varphi}
\title{Algebra I}
\author{Nicholas Schwab}
\date{Sommersemester 2017}

\newenvironment{alphanumerate}{\begin{enumerate}[label={\upshape(\alph*)}]}{\end{enumerate}}       
\newenvironment{rmnumerate}{\begin{enumerate}[label={\upshape(\roman*)}]}{\end{enumerate}}   
\newenvironment{diagram}{\begin{center}\begin{tikzpicture}}{\end{tikzpicture}\end{center}}  
                                          
\newtheoremstyle{cthm}
{\topsep}
{\topsep}
{\itshape}
{}
{\bfseries}
{.}
{ }
{\thmname{#1}\thmnumber{ \StrBehind{#2}{\thesubsection.}}\thmnote{ \textmd{(#3)}}\gdef\curthm{#2}}
\newtheoremstyle{cdef}
{\topsep}
{\topsep}
{}
{}
{\bfseries}
{.}
{ }
{\thmname{#1}\thmnumber{ \StrBehind{#2}{\thesubsection.}}\thmnote{ \textmd{(#3)}}\gdef\curthm{#2}}

\theoremstyle{cthm}
\newtheorem{lem}{Lemma}[subsection]
\newtheorem{thm}{Theorem}[subsection]
\newtheorem{sat}{Satz}[subsection]
\newtheorem{exc}{Exercise}[subsection]
\newtheorem{cor}{Corollary}[subsection]
\newtheorem{prop}{Proposition}[subsection]

\theoremstyle{cdef}
\newtheorem{defi}{Definition}[subsection]
\newtheorem{example}{Example}[subsection]
\newtheorem{rem}{Remark}[subsection]

\newcommand{\lbl}[1]{
	\label{#1}
	\edef\dummy{\curthm}
	\expandafter\xdef\csname thmref#1\endcsname{\dummy}
}

\newcommand{\reff}[1]{%
	\def\temp{\csname thmref#1\endcsname}%
	\StrBehind{\temp}{\thesubsection.}[\tempcropped]%
	\IfBeginWith{\temp}{\thesubsection}{\hyperref[#1]{\tempcropped}}{\hyperref[#1]{\temp}}%
}

\newcommand{\IN}{\mathbb{N}}
\newcommand{\IZ}{\mathbb{Z}}
\newcommand{\IQ}{\mathbb{Q}}
\newcommand{\IR}{\mathbb{R}}
\newcommand{\IC}{\mathbb{C}}

\renewcommand{\AA}{\mathfrak A}
\newcommand{\BB}{\mathfrak B}
\newcommand{\CC}{\mathfrak C}

\newcommand{\Hom}{\operatorname{Hom}}
\newcommand{\Ann}{\operatorname{Ann}}
\newcommand{\End}{\operatorname{End}}
\newcommand{\Aut}{\operatorname{Aut}}
\newcommand{\Gal}{\operatorname{Gal}}

%\newcommand{\multiline}[1]{#1}
\newcommand{\longto}{\longrightarrow}
\newcommand{\longot}{\longleftarrow}
\newcommand{\isomorphism}{
	\tikz[baseline=(a.base)] \node (a) at (0,0) {$\longrightarrow$} node[above=-0.25ex] {\tiny $\sim$};}
\newcommand{\morphism}[1][]{\overset{#1}{\longto}}

\newcommand{\ldotspam}{,\ldots,}
\newcommand{\st}{\ \middle|\ }
%\newcommand{\dashed}{dash pattern = on 2pt off 2pt}

\renewcommand{\phi}{\varphi}
\renewcommand{\epsilon}{\varepsilon}
\begin{document}

\maketitle
% start 2017-04-20
% organizational spam
\section{The Hilbert Basis- and Nullstellensatz}
\subsection{Noetherian Rings}
\begin{defi}\lbl{def:generatedIdeal}
 Let $R$ be a ring, and $f_1,\ldots, f_n\in R$ , then  the \emph{ideal generated by the $f_i$} is
 \begin{align*}\left( f_1,\ldots,  f_n\right)_R = \left\{\sum\lambda_i f_i\st\lambda_i \in R\right\} = \bigcap_{f_1,\ldots,f_n\in I\text{ ideal}} I\;.
 \end{align*}
The $f_i$ are called a \emph{basis} or \emph{generators} of $I$. 
\end{defi}
\begin{rem}
 If $I$ is not necessarily finite, 
 \begin{align*}
\left( f_i\st i\in I\right)_R = \left\{\sum_{i\in I} \lambda_i f_i \st\lambda_i = 0 \text{ for all but finitely many } i\right\} = \bigcap_{(f_i)_{i\in I}\subseteq I} I\;.
\end{align*}
\end{rem}
\begin{defi}\lbl{def:zeroOfIdeal}
 Let $k$ be a field, $I\subseteq k[X_1,\ldots, X_n]$ an ideal, $\ell$ a field extension of $k$. Call $x\in \ell^n$ a \emph{zero} of $I$ iff $f(x_1,\ldots,x_n) = 0$ for all $f\in I$. 
\end{defi}
\begin{rem}
 An element $x$ is a common zero of the $f_i\in k[X_1,\ldots,X_n]$ iff it is a zero of the ideal generated by the $f_i$.
\end{rem}
\begin{prop}\lbl{prop:Noetherian}
 For a ring $R$ the following conditions are equivalent:
 \begin{rmnumerate}
  \item Every ideal has a finite set of generators (i.e. is finitely generated).
  \item Every ascending chain $I_0 \subseteq I_1 \subseteq \ldots$ of ideals in $R$ terminates after finitely many steps, i.e. there is some $N\in\IN$ such that $I_n=I_N$ for all $n\geq N$.
  \item Every non-empty set $\mathfrak{M}$ of ideals in $R$ has an $\subseteq$-maximal element $I$. 
 \end{rmnumerate}
\end{prop}


\begin{defi}\lbl{def:Noetherian}
 A ring with these properties is called \emph{Noetherian}.
\end{defi}
\begin{example}
 Fields and principal ideal domains are Noetherian. 
\end{example}
\begin{thm}[Hilbert's Basissatz]\lbl{thm:Basissatz}
 If $R$ is Noetherian, so is $R[X_1,\ldots,X_n]$.
\end{thm}
\begin{cor}[of the Basissatz]
 Every polynomial system of equations in finitely many variables over a field has finite subsystem with the same set of solutions.
\end{cor}
\begin{thm}[Hilbert's Nullstellensatz] \lbl{thm:Nullstellensatz}
 Let $k$ be a algebraically closed field and $I$ be a proper ideal of $k[X_1,\ldots,X_n]$. Then $I$ has a zero $x\in k^n$.
\end{thm}
Both Hilbert's Nullstellensatz and Hilbert's Basissatz will be proved later on.
\subsection{Modules over rings}
\begin{defi}\lbl{def:module}
 An $R$-Module (where $R$ is a ring) is an abelian group $(M,+)$ with an operation
 \begin{align*}
  \cdot: R\times M \longto M\;,\quad  (r,m) \longmapsto r\cdot m
 \end{align*}
 such that for all $r,s\in R$ and $m,n\in M$
 \begin{align*}
  r\cdot(s\cdot m) &= (r\cdot s)\cdot m & (r+s)\cdot m &= r\cdot m + s\cdot m\\
  1\cdot m &= m & r\cdot(m+n)&= r\cdot m +r\cdot n\;. 
 \end{align*}
A \emph{morphism} of $R$-Modules is a map $M \overset{f}{\longto} N$ which is a homomorphism of abelian groups compatible with $\cdot$.
A \emph{submodule} of $M$ is a subgroup $X\subseteq M$ of $(M,+)$ such that $R\cdot X \subseteq X$. 
\end{defi}
\begin{example} The $R$-submodules of $R$ are the ideals in $R$.
\end{example}
\begin{prop} If $N\subseteq M$ is a $R$-submodule of the $R$-module $M$ the quotient group $M/N$ has a unique structure of an $R$-submodule such that the projection $M\overset{\pi}{\longto} M/N$ is a morphism of $R$-modules, and for arbitrary $R$-modules $T$ the map  
\begin{align*}
 \Hom_R(M/N, T) &\longto \left\{\tau\in \Hom_R(M,T)\st \tau|_N = 0\right\}\\
 t &\longmapsto \tau = t \circ \pi
\end{align*}
is bijective, where $t$ is surjective iff $\tau$ is and $t$ is injective iff $\ker(\tau)$ equals $N$.
\end{prop}
\begin{cor}
 Let $N,L\subseteq M$ be submodules of some $R$-Module $M$.
 \begin{rmnumerate}
 	\item There is a unique isomorphism $L/(N\cap L)\isomorphism (N+L)/N$ such that the following diagram commutes:
 	\begin{diagram}
 		\node (a) at (0,1.5) {$L$};
 		\node (b) at (3,1.5) {$N+L$};
 		\node (c) at (0,0) {$L/(N\cap L)$};
 		\node (d) at (3,0) {$(N+L)/N$};
 		\scriptsize
 		\draw[right hook->] (a) -- (b);
 		\draw[->] (a) -- (c) node[left,pos=0.5] {$\pi_{L/(N\cap L)}$};
 		\draw[->] (b) -- (d) node[right,pos=0.5] {$\pi_{(N+L)/N}$};
 		\draw[->, dashed] (c) -- (d) node[pos=0.5, above] {$\sim$};
 	\end{diagram}
 	\item If further $L\subseteq N$, there is a unique isomorphism  	 $M/N\isomorphism(M/L)/(N/L)$ such that the following diagram commutes:
 	 \begin{diagram}
 	 	\node (a) at (0,1.5) {$M$};
 	 	\node (b) at (3,1.5) {$M/L$};
 	 	\node (c) at (0,0) {$M/N$};
 	 	\node (d) at (3,0) {$(M/L)/(N/L)$};
 	 	\scriptsize
 	 	\draw[->] (a) -- (b) node[pos=0.5, above] {$\pi_{M/L}$};
 	 	\draw[->] (a) -- (c) node[pos=0.5, left] {$\pi_{M/N}$};
 	 	\draw[->] (b) -- (d) node[pos=0.5, right] {$\pi_{(M/L)/(N/L)}$};
 	 	\draw[->, dashed] (c) -- (d) node[pos=0.5, above] {$\sim$};
 	 \end{diagram}
 \end{rmnumerate}
\end{cor}
\begin{defi}
 If $M$ and $N$ are $R$-modules, $M\oplus N = M\times N$ equipped with component-by-component addition and scalar multiplication. This can be generalized to finitely many summands.
\end{defi}
\begin{example} $R^n =\left\{(r_i)_{i=1}^n\st r_i\in R\right\}$ is an $R$-module.
\end{example}
\begin{defi}\lbl{def:generatedModule}
 If $M$ is an $R$-module and $m_1,\ldots,m_k\in M$, then the \emph{submodule generated by $\{m_1\ldotspam m_k\}$ is}
 \begin{align*}
 \left\langle m_1\ldotspam m_k\right\rangle_R=Rm_1+\ldots+Rm_k=\left\{\sum r_i\cdot m_i\st r_i\in R\right\} = \bigcap_{m_1,\ldots,m_k\in X\text{ submodule}} X
 \end{align*}
As was the case for Definition \reff{def:generatedIdeal}, this can be generalized to infinitely many generators. $M$ is \emph{finitely generated} iff there are $m_1,\ldots,m_k\in M$ such that the submodules of $M$ generated by the $m_i$ equals $M$.
\end{defi}
\begin{prop} \lbl{prop:finitelyGeneratedSubmodules}
 Consider an exact sequence
 \begin{align*}
 	0\longto N\longto M\longto L\longto 0
 \end{align*}
 of $R$-modules.
 \begin{rmnumerate}
  \item If $M$ is finitely generated, then so is $L$.
  \item If $N$ and $L$ are finitely generated, then so is $M$.
 \end{rmnumerate}
 \end{prop}
 
\begin{cor}
 $M\oplus N$ is finitely generated iff $M$ and $N$ are. 
\end{cor}
\begin{prop}\lbl{prop:NoetherianModule}
Let $M$ be an $R$-module. The following properties are equivalent:
\begin{alphanumerate}
 \item Every submodule $N\subseteq M$ of $M$ is finitely generated.
 \item Every ascending sequence $N_0\subseteq N_1\subseteq \ldots$ of submodules of $N$ terminates.
 \item Every non-empty set $\mathfrak{M}$ of $R$-submodules of $M$ has a $\subseteq$-maximal element.
\end{alphanumerate}
\end{prop}
\begin{proof}
\begin{description}
 \item [$\mathbf{(a)\to (b)}$] Let $N_\infty = \bigcup_{i=0}^\infty N_i$, then this is a submodule, hence finitely generated by a). Let $n_1,\ldots, n_k$ generate $N_\infty$. Choose $\ell_i$ such that $n_i\in N_{\ell_i}$ and let $\ell = \max_{i\leq k}\ell_i$, then $N_\ell = N_\infty$.
 \item [$\mathbf{(b)\to (c)}$] From b) we conclude, that in the $\subseteq$-ordered set $\mathfrak{M}$ every ascending chain has an upper bound in $\mathfrak{M}$, namely the ideal, that terminates the chain. Therefore by Zorn's Lemma there is $\subseteq$-maximal element in $\mathfrak{M}$.
 \item[$\mathbf{(c)\to (a)}$] Let $\mathfrak{M}$ be the set of finitely generated submodules of $N$. Since $\{0\}\subseteq N$ is a module, this set is not empty. Therefore there is a $\subseteq$-maximal submodule $P$ in $\mathfrak{M}$ generated by $p_1,\ldots, p_n$. Therefore there is no $f\in N\setminus P$ such that $\langle p_1,\ldots, p_n, f\rangle_R$ is a submodule of $N$ since this would be a superset of $P$. Hence we have $N=P$ is finitely generated.
\end{description}
\end{proof}

% end 2017-04-20
% start 2017-04-24
\begin{defi}\lbl{def:NoetherianModule}
 A module over a ring $R$ is \emph{Noetherian} iff the equivalent conditions above are fulfilled.
\end{defi}
\begin{rem}\lbl{rem:subQuotientNoetherian}
 Sub- and quotient modules of Noetherian rings are Noetherian. If $N$ is a submodule of $M$ and if $N$ and $M/N$ are Noetherian, then $M$ is Noetherian.
\end{rem}
\begin{proof}
 The first assertion follows easily from Proposition \reff{prop:finitelyGeneratedSubmodules} and the characterization of \emph{Noetherian modules} by Proposition \reff{prop:NoetherianModule}(a). For the second assertion let $N$ and $M/N$ be Noetherian and $X\subseteq M$ be a submodule. Since both $(X\cap N)\subseteq N$ and $X/(X\cap N)\simeq(X+N)/N\subseteq M/N$ are finitely generated as submodules of $N$, $M/N$ respectively, we obtain the exact sequence 
 \begin{align*}
 	0\longto X\cap N\longto X\longto X/(X\cap N)\longto 0\;,
 \end{align*}
 proving that $X$ is finitely generated by Proposition \reff{prop:finitelyGeneratedSubmodules}. 
\end{proof}
\begin{rem}
 Any Noetherian module is finitely generated.
\end{rem}
\begin{prop}\lbl{prop:ringNoetherianModule}
Let $R$ be a Noetherian ring. Then any finitely generated $R$-module is Noetherian.
\end{prop}
\begin{proof}
	We proceed by induction on the number of generators of $M$. The case of only one generator is immediate. Now let $M=Rm_1+\ldots+Rm_k$ and any $R$y -module with less than $k$ generators be Noetherian. In particular, $N=Rm_1+\ldots+Rm_{k-1}$ is Noetherian. The map $R\to M/N$ sending $r\in R$ to $rm_k+N$ is surjective, hence $M/N$ is isomorphic to some quotient of $R$ and thus Noetherian by Remark \reff{rem:subQuotientNoetherian}. Then, again by Remark \reff{rem:subQuotientNoetherian}, $M$ is Noetherian.\end{proof}
\begin{defi}\lbl{def:annihilator}
 For a module $M$ over a ring $R$, define 
 \begin{align*}
 	\Ann(M)=\{r\in R\mid r\cdot M = \{0\}\} = \{r\in R\mid r\cdot m = 0\ \forall m\in M\}\;.
 \end{align*}
 It is called the \emph{annihilator} or \emph{annulator} of $M$.
\end{defi}
\begin{prop}
 A module $M$ over a ring $R$ is Noetherian iff it is finitely generated and $R/\Ann(M)$ is a Noetherian ring.
\end{prop}

\subsection{Proof of the Hilbert basis theorem}\lbl{sec:HilbertBasisProof}
\begin{proof}\lbl{proof:HilbertBasis}
Let $R$ be a Noetherian ring and $I\subseteq R[T]$ be an ideal. Let $R[T]_{\leq n}$ be the set of polynomials over $R$ of degree smaller or equal to $n$. This is isomorphic to $R^{n+1}$ ($1,\ldots, T^n$ being free generators) as $R$-modules, thus Noetherian (Proposition \reff{prop:ringNoetherianModule}) which implies that $I_{\leq n} = I \cap R[T]_{\leq n}$ is a finitely generated $R$-module. Let $I_n$ be the set of all $a_n\in R$, such that $a_0+a_1T+\ldots+a_nT^n\in I$ for some $a_0,\ldots,a_{n-1}\in R\}$. This is an ideal ($R$-submodule) of $R$, being the image of $I_{\leq n} \to R$ sending $a_0+a_+\ldots+a_nT^n\in I_{\leq n}$ to $a_n$. We have $I_n\subseteq I_{n+1}$ as $T\cdot I_{\leq n}\subseteq I_{\leq n+1}$. As $R$ is Noetherian, this chain terminates at some $N\in\IN$ with $I_n = I_N$ for $n\geq N$. Let $f_1,\ldots, f_k$ be generators of $I_{\leq N}$ as an $R$-module. We claim that they generate $I$ as an $R[T]$-module. Since they generate $I_{\leq N}$ as an $R$-module, their $N$-th coefficients $f_N^{(i)}$, where $i\leq k$, generate $I_n = I_N$, for $n\geq N$, as an $R$-module.

We show by induction on $n$, that any $g\in I_{\leq n}$ belongs to $\left(f_1,\ldots,f_k\right)_{R[T]}$, thus establishing $I= \left(f_1,\ldots, f_k\right)_{R[T]}$. For $n\leq k$ we have $g\in I_{\leq N}$ and the assertion is obvious. Let $n>N$ let the assertion be valid for all $h \in I_{\leq n-1}$. Let $g=\sum_{i=1}^n g_iT^i$, $g_n = \sum_{i=1}^k \gamma_i f_N^{(i)}$ and $h = g-\sum_{i=1}^k\gamma_i T^{n-N} f_i$, then $h\in I_{\leq n-1}$ as the coefficient of $T^n$ cancels. Thus, $h = \sum_{i=1}^k\rho_i f_i$ with $\rho_i\in R[T]$ by the induction assumption and
\begin{align*}
	g=\sum_{i=1}^k(\gamma_i T^{n-k} +\rho_i) f_i \in \left( f_1,\ldots,f_k\right)_{R[T]}
\end{align*}
as claimed. This shows that $I$ is finitely $R[T]$-generated, hence $R[T]$ is Noetherian.
\end{proof}
\begin{cor}\lbl{cor:NoetherianPolynomial}
 If $R$ is a Noetherian ring, so is $R[X_1,\ldots,X_n]$ for all $n\in\IN$.
\end{cor}
\subsection{Finiteness properties of $R$-algebras}
\begin{defi}
 Let $R$ be a ring. An \emph{$R$-algebra} is a ring $A$ (commutative, with 1) together with a ring homomorphism $R\overset{\alpha}{\longto} A$. Then $A$ becomes an $R$-module via $r\cdot a \coloneqq \alpha(r) \cdot a$. We call $A$ \emph{finite over $R$} (or \emph{finite as an $R$-algebra}) if it is finitely generated as an $R$-module. We call $A$ of \emph{finite type over $R$} if it is finitely generated as an $R$-algebra in the sense that there are $f_1,\ldots, f_k\in A$, $k\in \IN$, such that any $R$-subalgebra $B\subseteq A$ (i.e. any subring $B\subseteq A$ which is also a $R$-submodule, or, equivalently, a subring containing the image of $\alpha$) containing the $f_i$ must equal $A$.
\end{defi}
\begin{rem}\lbl{rem:generatedSubalgebra}
 If $A$ is an $R$-algebra and $f_1,\ldots,f_k\in A$, the following subsets of $A$ coincide:
 \begin{itemize}
  \item $\left\{\sum r_\alpha f_1^{\alpha_1}\cdot\ldots\cdot f_k^{\alpha_k}\st r_\alpha\in R, r_\alpha\neq 0 \text{ only for finitely many } \alpha\right\}$
  \item The image of the ring homomorphism $R[X_1,\ldots,X_k]\to A$ sending $p\in R[X_1,\ldots, X_k]$ to $p(f_1,\ldots,f_k)$.
  \item The intersection of all $R$-subalgebras of $A$ containing the $f_i$.
 \end{itemize}
Thus, an $R$-algebra $A$ is of finite type iff it is isomorphic to a quotient of $R[X_1,\ldots, X_k]$ by some ideal $I$ for finite $k$.
\end{rem}
\begin{rem}
\begin{enumerate}[a)]
 \item Obviously, if $f_1,\ldots, f_i\in A$ generate $A$ as an $R$-module, they generate it as an $R$-algebra. Thus any finite $R$-algebra is of finite type. On the other side, when $R\neq \{0\}$ and and $n>0$, $R[X_1, \ldots, X_n]$ is an $R$-algebra of finite type that is not finitely generated as an $R$-module.
\item Obviously, if $L/K$ is a field extension then $L$ is a finite $K$-algebra iff the field extension is finite. The fact that this still holds if $L$ is a $K$-algebra of finite type turns out to be essentially equivalent to the Nullstellensatz.
 \end{enumerate}

\end{rem}


\begin{prop}\lbl{prop:1.4.1}
 Let $R$ be a ring, $A$ an $R$-algebra. Any $A$-algebra $B$ becomes an $R$-algebra via the composition $R\to A\to B$.
 \begin{rmnumerate}
  \item If $A$ is finite over $R$, it is of finite type over $R$.
  \item (transitivity of finiteness) If $B$ is finite over $A$ and $A$ finite over $R$, then $B$ is finite over $R$.
  \item If $B$ over $A$ and $A$ over $R$ are of finite type, then $B$ is of finite type over $R$.
  \item An algebra of finite type over a Noetherian ring is a Noetherian ring.
 \end{rmnumerate}

\end{prop}
%end 2017-04-24
%start 2017-04-27
\begin{proof}
 \begin{rmnumerate}
  \item Trivial.
  \item If $b_1,\ldots,b_m$ generate $B$ as an $A$-module and $a_1,\ldots,a_n$ generate $A$ as an $R$-module, the $\beta_{i,j} = a_j\cdot b_i$ generate $B$ as an $R$-module: Indeed, let $b\in B$, then $b = \sum_{i=1}^m \alpha_i b_i$ (with $\alpha_i\in A$) and each $\alpha_i$ can be written as $\alpha_i = \sum_{j=1}^n r_{i,j}a_j$. Then $b = \sum_{i=1}^m \sum_{j=1}^n r_{i,j} \beta_{i,j}$.
  \item By Remark \reff{rem:generatedSubalgebra}, we obtain surjective homomorphisms $A[Y_1,\ldots,Y_m]\morphism[\beta]B$ (as $A$-algebras, hence also as $R$-algebras) and $R[X_1,\ldots,X_n]\morphism[\alpha]A$ (as $R$-algebras). Lifting the latter to a surjective homomorphism $R[X_1,\ldots,X_n,Y_1,\ldots,Y_m]\to A[Y_1,\ldots,Y_m]$ and composing them provides us with a surjective homomorphism
  \begin{align*}
  	R[X_1,\ldots,X_n,Y_1,\ldots,Y_m]\longto B\;,
  \end{align*}
  proving that $B$ is of finite type over $R$. In particular, if $b_1,\ldots,b_m$ generate $B$ as an $A$-algebra and $a_1,\ldots,a_n$ generate $A$ as an $R$-algebra, then $B$ is generated by $a_1,\ldots, a_n, b_1,\ldots, b_m$ as an $R$-algebra.
  \item Note that the quotient of a Noetherian ring by an ideal stays Noetherian: The preimage of an infinitely ascending chain of ideals of the quotient ring would be an infinitely ascending chain of ideals of the original ring. Now if $a_1\ldotspam a_m\in A$ generate $A$ as an $R$-algebra, then
  \begin{align*}
   R[X_1\ldotspam X_m] &\longto A\\
   p&\longmapsto p(a_1\ldotspam a_m)
  \end{align*}
  is surjective and $A$ is isomorphic to a quotient of $R[X_1\ldotspam X_m]$, which by the Basissatz is Noetherian if $R$ is.
 \end{rmnumerate}
 \end{proof}

 \begin{prop}[Artin-Tate]\lbl{prop:artinTate}
  Let $R$ be a Noetherian ring, $A$ an $R$-algebra of finite type and $B\subseteq A$ an $R$-subalgebra such that $A$ is finite over $B$. Then $B$ is an $R$-algebra of finite type.
 \end{prop}
 \begin{proof}
 Let $a_1,\ldots,a_m$ generate $A$ as an $R$-algebra and let $\alpha_1,\ldots,\alpha_n$ generate it as a $B$-module. We have expressions
 \begin{align}\lbl{eq:Prop142*}
  a_i =\sum_{j=1}^n b_{i,j} \alpha_j\quad\text{and}\quad
  \alpha_k\cdot \alpha_k = \sum_{j=1}^n \beta_{j,k,l} \alpha_j.\tag{$*$}
 \end{align}
Let $\BB\subseteq B$ be the $R$-algebra generated by the $b_{i,j}$ and the $\beta_{j,k,l}$. It is of finite type over $R$ thus Noetherian. Let $\AA \subseteq A$ be the $\BB$-submodule generated by $\alpha_1,\ldots,\alpha_n$. It is a subring containing the $a_i$ by (\reff{eq:Prop142*}) and is an $R$-algebra because $\BB$ is. Then $\AA=A$ and $A$ is finite over $\BB$. Since $\BB$ is Noetherian, $B\subseteq A$ is a $\BB$-subalgebra, and $B$ is finitely generated as $\BB$-module ($\BB$ being Noetherian), $B$ is of finite type over $\BB$ (Proposition \reff{prop:1.4.1}(i)) and thus also over $R$ (Proposition \reff{prop:1.4.1}(iii)).
\end{proof}
\begin{prop}[Eakin-Nagata]\lbl{prop:eakinNagata}
 Let $A$ be a Noetherian ring and $B\subseteq A$ be a subring such that $A$ is finite over $B$. Then $B$ is Noetherian.
\end{prop}
\begin{rem}
 See Matsumura, CRT, for Eakin-Nagata.
\end{rem}
\subsection{The notion of integrity and the Noether Normalization Theorem}
Remark of the author: It's called integrity not entireness...
\begin{defi}\lbl{def:integrity}
 Let $A\subseteq B$ be a ring extension. We call $b\in B$ \emph{integral} over $A$ if it satisfies an equation
 \begin{align*}
  b^n +a_{n-1}b^{n-1}+\ldots+a_1b+a_0 =0
 \end{align*}
 with $a_0,\ldots,a_{n-1}\in A$. We call $B$ over $A$ \emph{integral}, if every element of $B$ is integral.
\end{defi}
\begin{rem}
 It is not really necessary to assume $A\to B$ to be injective.
\end{rem}
\begin{prop}
 \begin{rmnumerate}
  \item An element $b\in B$ is integral over $A$ iff there is an intermediate ring $A\subseteq C\subseteq B$ containing $b$ which is finite over $A$. If $b_1\ldotspam b_n$ are finitely many integral elements of $B$, there is an $A$-subalgebra $A\subseteq C\subseteq B$ containing all $b_i$ which is finite over $A$.
  \item The elements of $B$ which are integral over $A$ form a subring of $B$, the \emph{integral closure} of $A$ in $B$.
  \item If $C/B$ and $B/A$ are integral, so is $C/A$.
  \item Let $B/A$ be integral (where it is essential that $A$ is a subring of $B$). If $B$ is a field, then so is $A$.
 \end{rmnumerate}

\end{prop}
\begin{proof}
 \begin{rmnumerate}
  \item Let $b_1\ldotspam b_n$ be integral over $A$. Each $b_i$ satisfies an equation
  \begin{align*}
  	b_j^{d_i}=\sum_{i=0}^{d_i-1}a_{i,j}b_j^i\quad\text{where }a_{i,j}\in A\;.
  \end{align*}
  Then the subring $C=A[b_1,\ldots,b_n]$ is generated by all $b_1^{k_1}\cdots b_n^{k_n}$ where $0\leq k_i<d_i$, hence it is finite over $A$. The first assertion of (i) follows as a special case.
  
   For the other direction let $C\subseteq B$ be an $A$-subalgebra which is finitely generated as an $A$-module, say, by $\gamma_1,\ldots,\gamma_n$. Let $b\in C$ and choose $m_{i,j}\in A$ such that
   \begin{align*}
   	b\gamma_j=\sum_{i=1}^n m_{i,j} \gamma_j
   \end{align*}
   The matrix $M=(m_{i,j})_{i,j=1}^n$ satisfies its own characteristic equation by Cayley-Hamilton theorem: $M^n = p_0+p_1M+\ldots+p_{n-1}M^{n-1}$ for suitable $p_0,\ldots,p_{n-1}\in A$. Since $b^j$ in $C$ can be expressed by $M^j$ (in the sense that 
  \begin{diagram}
 	\node (a) at (0,1.5) {$A^n$};
 	\node (b) at (2,1.5) {$A^n$};
 	\node (c) at (0,0) {$C$};
 	\node (d) at (2,0) {$C$};
 	\scriptsize
 	\draw[->] (a) -- (b) node[pos=0.5,above] {$M^j\cdot$};
 	\draw[->>] (a) -- (c) node[pos=0.5, left] {$\gamma$};
 	\draw[->>] (b) -- (d) node[pos=0.5,right] {$\gamma$};
 	\draw[->] (c) -- (d) node[pos=0.5, above] {$\cdot b^j$};
 	\footnotesize
 	\node (a1) at (-1.5,1.5) {$(a_1,\ldots,a_n)$};
 	\node (c1) at (-1.5,0) {$\sum a_i\gamma_i$};
 	\draw[|->] (a1) -- (c1);
 	\node (b1) at (3.5,1.5) {$(a_1,\ldots,a_n)$};
 	\node (d1) at (3.5,0) {$\sum a_i\gamma_i$};
 	\draw[|->] (b1) -- (d1);
  \end{diagram}
  commutes) it follows, that $b^n \cdot c = p_0c+p_1bc+\ldots+p_{n-1}b^{n-1}c$ (first for $c=\gamma_i$, then all $c\in C$). Taking $c=1$ shows that $b$ is indeed integral over $A$.
  \item If $C$ is as in $A$ and contains $b_1, b_2$, then it contains $b_1\pm b_2$ and $b_1\cdot b_2$, showing that these are integral over $A$. 
  \item Let, more generally, $B/A$ be integral and $c\in C$ integral over $B$. It satisfies an equation $c^d = \beta_0+\beta_1c+\ldots+\beta_{d-1}c^{d-1}$ with $\beta_i\in B$. By (i), there is an $A$-subalgebra $\BB\subseteq B$ which is finite over $A$ and contains the $\beta_i$. Then $c$ is integral over $\BB$, hence by (i) there is a $\BB$-subalgebra $\CC\subseteq C$ containing $c$ and finite over $\BB$. Now $\CC/A$ is finite by Proposition \reff{prop:1.4.1}(ii), hence $c$ is integral over $A$ by (i).
 \end{rmnumerate}
 

\end{proof}
%end 2017-04-27



\end{document}
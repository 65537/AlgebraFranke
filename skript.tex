\documentclass[DIV=14,parskip=half]{scrartcl}
\usepackage[utf8]{inputenc}
\setkomafont{sectioning}{\rmfamily\bfseries}

\usepackage{amsmath,amsfonts,amssymb,mathtools}
\usepackage{amsthm}
\usepackage{bm}
\usepackage{tikz}
\usepackage[shortlabels]{enumitem}

\usepackage[english]{babel}
\usetikzlibrary{calc}
\usepackage{subcaption}

\renewcommand{\phi}{\varphi}
\title{Algebra I}
\author{Nicholas Schwab}
\date{Sommersemester 2017}
\newenvironment{alphanumerate}{\begin{enumerate}[a)]}{\end{enumerate}}                                                     
\newtheorem{lem}{Lemma}[subsection]
\newtheorem{thm}{Theorem}[subsection]
\newtheorem{sat}{Satz}[subsection]
\newtheorem{exc}{Exercise}[subsection]
\newtheorem{cor}{Corollary}[subsection]
\newtheorem{prop}{Proposition}[subsection]
\theoremstyle{definition}
\newtheorem{defi}{Definition}[subsection]
\newtheorem{example}{Example}[subsection]
\newtheorem{rem}{Remark}[subsection]

\newcommand{\N}{\mathbb{N}}
\newcommand{\Z}{\mathbb{Z}}
\newcommand{\Q}{\mathbb{Q}}
\newcommand{\R}{\mathbb{R}}
\newcommand{\C}{\mathbb{C}}
\newcommand{\Hom}{\operatorname{Hom}}
\newcommand{\multiline}[1]{#1}
\newcommand{\longto}{\longrightarrow}
\newcommand{\longot}{\longleftarrow}
\newcommand{\Ann}{\operatorname{Ann}}
\newcommand{\ldotspam}{,\ldots,}

\renewcommand{\phi}{\varphi}
\renewcommand{\epsilon}{\varepsilon}
\begin{document}

\maketitle
% start 2017-04-20
% organizational spam
\section{The Hilbert Basis- and Nullstellensatz}
\subsection{Noetherian Rings}
\begin{defi}\label{def:generatedIdeal}
 Let $R$ be a ring, and $f_1,\ldots, f_n\in R$ , then 
 \begin{align*}\langle f_1,\ldots,  f_n\rangle_R = \left\{\sum_{i=1}^n \lambda_i f_i \middle| \lambda_i \in R\right\} = \bigcap_{\substack{I\subseteq R,\\ I \text{ ideal },\\ f_i\in I\forall i}} I.
 \end{align*}
This is called the \emph{ideal} generated by the $f_i$ and the $f_i$ are called a \emph{basis} or \emph{generators} of $I$. 
\end{defi}
\begin{rem}
 If $I$ is not necessarily finite, 
 \begin{align*}
\langle f_i\mid i\in I\rangle_R = \left\{\sum_{i\in I} \lambda_i f_i \middle| \lambda_i = 0 \text{ for all but finitely many } i\right\} = \bigcap_{\substack{I\subseteq R,\\ I \text{ ideal },\\ f_i\in I\forall i}} I.
\end{align*}
\end{rem}
\begin{defi}\label{def:zeroOfIdeal}
 Let $k$ be a field, $I\subseteq k[T_1,\ldots, T_n]$ an ideal, $l$ a field extension of $k$. $x\in l^n$ is a zero of $I$ iff $f(x_1,\ldots,x_n) = 0$ for all $f\in I$. 
\end{defi}
\begin{rem}
 $x$ is a common zero of the $f_i\in k[X_1,\ldots,X_n]$ iff is a zero of the ideal generated by the $f_i$.
\end{rem}
\begin{prop}\label{prop:Noetherian}
 For a ring $R$ the following conditions are equivalent:
 \begin{enumerate}[a)]
  \item Every ideal has a finite set of generators (i.e. is finitely generated).
  \item Every ascending chain $I_0 \subseteq I_1 \subseteq \ldots$ of ideals in $R$ terminates after finitely many steps, i.e. there is some $n\in\N$ such that $I_k=I_n$ for all $k\geq n$.
  \item Every non-empty set $\mathfrak{M}$ of ideals in $R$ has an $\subseteq$-maximal element $I$. 
 \end{enumerate}

\end{prop}


\begin{defi}\label{def:Noetherian}
 A ring with these properties is called \emph{Noetherian}.
\end{defi}
\begin{example}
 Fields and principal ideal domains are Noetherian. 
\end{example}
\begin{thm}[Hilbert's Basissatz]\label{thm:Basissatz}
 If $R$ is Noetherian, $R[T_1,\ldots,T_n]$ (with finite $n$!) is Noetherian.
\end{thm}
\begin{proof}
 The proof is recapitulated later on.
\end{proof}
\begin{cor}[of the Basissatz]
 Every polynomial system of equations in finitely many variables over a field has finite subsystem with the same set of solutions.
\end{cor}
\begin{thm}[Hilbert's Nullstellensatz] \label{thm:Nullstellensatz}
 Let $k$ be a algebraically closed field and $I$ be a proper ideal of $k[X_1,\ldots,X_n]$. Then $I$ has a zero $x\in k^n$.
\end{thm}
\begin{proof}
 This will be proofed in a few days.
\end{proof}
\subsection{Modules over rings}
\begin{defi}\label{def:module}
 An $R$-Module (where $R$ is a ring) is an abelian group $(M,+)$ with an operation
 \begin{align*}
  \cdot: R\times M &\longto M\\
  (r,m) &\longmapsto r\cdot m
 \end{align*}
 such that
 \begin{align*}
  r\cdot(s\cdot m) &= (r\cdot s)\cdot m \\
  (r+s)\cdot m &= r\cdot m + s\cdot m\\
  r\cdot(m+n) &= r\cdot m +r\cdot n\\
  1\cdot m &= m.
 \end{align*}
A morphism of $R$-Modules is a map $M \overset{f}{\longto} N$ which is a homomorphism of abelian groups compatible with $\cdot$.
A submodule of $M$ is a subgroup $X\subseteq M$ of $(M,+)$ such that $R\cdot X \subseteq X$. 
\end{defi}
\begin{example} The $R$-submodules of $R$ are the ideals in $R$.
\end{example}
\begin{prop} If $N\subseteq M$ is a $R$-submodule of the $R$-module $M$ the quotient group $M/N$ has a unique structure of an $R$-submodule such that the projection $M\overset{\pi}{\longto} M/N$ is a morphism of $R$-modules, and for arbitrary $R$-modules $T$ the map  
\begin{align*}
 \Hom_R(M/N, T) &\longto \{\tau\in \Hom_R(M,T)| \tau|_N = 0\}\\
 t &\longmapsto \tau = t \circ \pi
\end{align*}
is bijective, where $t$ is surjective iff $\tau$ is and $t$ is injective iff $\ker(\tau)$ equals $N$.
\end{prop}
\begin{rem}
 Two important corollaries are:
 \begin{align*}
  (M/L)/(N/L) \overset{\simeq}{\longleftarrow} M/N
 \end{align*}
for $M\supseteq N \supseteq L$ and, for submodules $N$ and $L$ of $M$
\begin{align*}
 (N+L)/N \overset{\simeq}{\longot} L/(N\cap L)
\end{align*}
where $N+L$ denotes the submodule $\{l+n|l\in L, n\in N\}$ of $M$.
\end{rem}
\begin{defi}
 If $M$ and $N$ are $R$-modules, $M\oplus N = \{(m,n),|m\in M, n\in N\} = M\times N$ equipped with component-by-component addition and scalar multiplication. This can be generalized to finitely many summands.
\end{defi}
\begin{example} $R^n = \{(r_i)_{i=1}^n |r_i\in R\}$ is an $R$-module.
\end{example}
\begin{defi}\label{def:generatedModule}
 If $M$ is an $R$-module and $m_1,\ldots,m_k\in M$, then the submodule generated by $\{m_i|1\leq i\leq k\}$ is
 \begin{align*}
  \left\{\sum_{i=1}^k r_i\cdot m_i \middle| r_i\in R\right\} = \bigcap_{\substack{X\subseteq M\\ X \text{ module}\\ \text{all } m_i\in X}} X
 \end{align*}
As was the case for Definition \ref{def:generatedIdeal}, this can be generalized to infinitely many generators. $M$ is finitely generated iff there are $(m_i)_{i=1}^k$, $k\in \N$, $m_i\in M$ such that the submodules of $M$ generated by the $m_i$ equals $M$.
\end{defi}
\begin{prop} \label{prop:finitelyGeneratedSubmodules}
 Let $N\subseteq M$ be an $R$-submodule
 \begin{enumerate}[a)]
  \item If $M$ is finitely generated, $M/N$ is finitely generated.
  \item If $N$ and $M/N$ are finitely generated, $M$ is finitely generated.
 \end{enumerate}
 \end{prop}
 
\begin{cor}
 $M\oplus N$ is finitely generated iff $M$ and $N$ are. (Note that: $M\simeq M\oplus\{0\}$ and $(M\oplus N)/M \simeq N$)
\end{cor}
\begin{prop}\label{prop:NoetherianModule}
Let $M$ be an $R$-module. The following properties are equivalent:
\begin{enumerate}[a)]
 \item Every submodule $N\subseteq M$ of $M$ is finitely generated.
 \item Every ascending sequence $N_0\subseteq N_1\subseteq \ldots$ of submodules of $N$ terminates.
 \item Every non-empty set $\mathfrak{M}$ of $R$-submodules of $M$ has a $\subseteq$-maximal element.
\end{enumerate}
\end{prop}
\begin{proof}
\begin{description}
 \item [$\mathbf{a)\to b)}$] Let $N_\infty = \bigcup_{i=0}^\infty N_i$, then this is a submodule, hence finitely generated by a). Let $n_1,\ldots, n_k$, $k\in\N$, generate $N_\infty$ and let $j_i$, for $1\leq i \leq k$, be chosen such that $n_i\in N_{j_i}$ and let $l = \max\{j_i|1\leq i \leq k\}$, then $n_l = N_\infty$.
 \item [$\mathbf{b)\to c)}$] From b) we conclude, that in the $\subseteq$-ordered set $\mathfrak{M}$ every ascending chain has an upper bound in $\mathfrak{M}$, namely the ideal, that terminates the chain. Therefore by Zorn's Lemma there is $\subseteq$-maximal element in $\mathfrak{M}$.
 \item[$\mathbf{c)\to a)}$] Let $\mathfrak{M}$ be the set of finitely generated submodules of $N$. Since $\{0\}\subseteq N$ is a module, this set is not empty. Therefore there is a $\subseteq$-maximal submodule $P$ in $\mathfrak{M}$ generated by $p_1,\ldots, p_n$. Therefore there is no $f\in N\setminus P$ such that $\langle p_1,\ldots, p_n, f\rangle_R$ is a submodule of $N$ since this would be a superset of $P$. Hence we have $N=P$ is finitely generated.
\end{description}
\end{proof}

% end 2017-04-20
% start 2017-04-24
\begin{defi}\label{def:NoetherianModule}
 A module over a ring $R$ is \emph{Noetherian} iff the equivalent conditions above are fulfilled.
\end{defi}
\begin{rem}\label{rem:subQuotientNoetherian}
 Sub- and quotient modules of Noetherian rings are Noetherian. If $N$ is a submodule of $M$ and if $N$ and $M/N$ are Noetherian, then $M$ is Noetherian.
\end{rem}
\begin{proof}
 The first assertion follows easily from Proposition \ref{prop:finitelyGeneratedSubmodules} and the characterization of \emph{Noetherian modules} by Proposition \ref{prop:NoetherianModule}a). For the last assertion, let $N$ and $M/N$ be Noetherian and $X\subseteq M$ be a submodule. Then $X\cap N$ is a submodule of $N$, thus finitely generated, and $X/(X\cap N) \simeq (X+N)/N$ is isomorphic to a submodule of $M/N$, thus finitely generated and $X$ is finitely generated by Proposition \ref{prop:finitelyGeneratedSubmodules}. 
\end{proof}
\begin{rem}
 Any Noetherian module is finitely generated.
\end{rem}
\begin{prop}\label{prop:ringNoetherianModule}
 For a ring $R$ the following conditions are equivalent:
 \begin{enumerate}[a)]
  \item $R$ is Noetherian in the sense of definition \ref{def:Noetherian}.
  \item $R$ is Noetherian as $R$-module.
  \item Any finitely generated $R$-module is Noetherian.
 \end{enumerate}

\end{prop}
\begin{proof}
 \begin{description}
  \item [$\mathbf{a)\leftrightarrow b)}$] Follows from the definition.
  \item [$\mathbf{c)\to b)}$] Obvious, as $R$ is a finitely generated $R$-module.
  \item [$\mathbf{b)\to c)}$] Induction on the number of generators of $M$. Let $M$ be generated by $m_1,\ldots,m_k$ as an $R$-module and let $R$-modules generated by $<k$ elements be Noetherian, let $N= \sum_{i=1}^{k-1} R\cdot m_i = \left\{\sum_{i=1}^{k-1} \rho_i\cdot m_i |\rho_i \in R\right\}$ be the submodule generated by the first $k-1$ of the $m_i$. By the induction hypothesis, is is Noetherian. The map $R\longto M/N$ sending $r\in R$ to the image of $r\cdot m_k$ in $M/N$ is surjective. This, $M/N$ is isomorphic to a quotient of $R$, the Noetherian by Remark \ref{rem:subQuotientNoetherian}. Also by Remark \ref{rem:subQuotientNoetherian}, $M$ is Noetherian.
 \end{description}

\end{proof}
\begin{defi}\label{def:annihilator}
 For a module $M$ over a ring $R$, let $\Ann(M)$ be $\{r\in R\mid r\cdot M = \{0\}\} = \{r\in R\mid r\cdot m = 0 \forall m\in M\}$. It is called the \emph{annihilator} or \emph{annulator} (?) of $M$.
\end{defi}
\begin{prop}
 A module $M$ over a ring $R$ is Noetherian iff it is finitely generated and $R/\Ann(M)$ is a Noetherian ring.
\end{prop}

\subsection{Proof of the Hilbert basis theorem}\label{sec:HilbertBasisProof}
\begin{proof}\label{proof:HilbertBasis}
Let $R$ be a Noetherian ring and $I\subseteq R[T]$ be an ideal. Let $R[T]_{\leq n}$ be the set of polynomials over $R$ of degree smaller or equal to $n$. This is isomorphic to $R^{n+1}$ ($1,\ldots, T^n$ being free generators) as $R$-modules, thus Noetherian as an $R$-module (Proposition \ref{prop:ringNoetherianModule}) which implies that $I_{\leq n} = I \cap R[T]_{\leq n}$ is a finitely generated $R$-module. Let $I_n$ be $\{a_n|\sum_{i=0}^na_iT^i \in I, \text{ for some } a_0,\ldots,a_{n-1}\in R\}$. This is an ideal ($R$-submodule) of $R$, being the image of $I_{\leq n} \longto R$ sending $\sum_{i=0}^n\in I_{\leq n}$ to $a_n$. We have $I_n\subseteq I_{n+1}$ as $T\cdot I_{\leq n}\subseteq I_{\leq n+1}$. As $R$ is Noetherian this terminates at some $k\in\N$ with $I_n = I_k$ for $n\geq k$. Let $f_1,\ldots, f_A$ be generators of $I_{\leq k}$ as an $R$-module. We claim that they generate I as a $R[T]$-module. Since they generate $I_{\leq k}$ as an $R$-module, their $k$-th coefficients $f_{i,k}$, $1\leq i\leq A$, generate $I_n = I_k$, for $n\geq k$, as an $R$-module.

We show, by induction on $n$, that any $g\in I_{\leq n}$ belongs to $\langle f_1,\ldots,f_A\rangle_{R[T]}$, establishing $I= \langle f_1,\ldots, f_A\rangle_{R[T]}$. For $n\leq k$ we have $g\in I_{\leq k}$ and the assertion is obvious. Let $n>k$ let the assertion be valid for all $\tilde g \in I_{\leq n-1}$. Let $g=\sum_{i=1}^n g_iT^i$, $g_n = \sum_{i=1}^A \gamma_i f_{i,k}$, let $\tilde g = g-\sum_{i=1}^A \gamma_i T^{n-k} f_i$, then $\tilde g\in I_{\leq n}$ as the coefficients cancel. Thus, $\tilde g = \sum_{i=1}^A\rho_i f_i$ with $\rho_i\in R[T]$ by the induction assumption and $g=\sum_{i=1}^A(\gamma_i T^{n-k} +\rho_i) f_i = \langle f_1,\ldots,f_A\rangle_{R[T]}$ as claimed.

Thus $I$ is finitely $R[T]$-generated. Since this holds for any $I\subseteq R[T]$, $R[T]$ is Noetherian.
\end{proof}
\begin{cor}\label{cor:NoetherianPolynomial}
 As $R[X_1,\ldots,X_{n+1}] \simeq (R[X_1,\ldots,X_n])[X_{n+1}]$, it follows by induction that arbitrary finite polynomial rings over Noetherian rings are Noetherian.
\end{cor}
\subsection{Finiteness properties of $R$-algebras}
\begin{defi}
 Let $R$ be a ring. An \emph{$R$-algebra} is a ring $A$ (commutative, with 1) together with a ring homomorphism $R\overset{\alpha}{\longto} A$. The $A$ becomes an $R$-module by $r\cdot a \coloneqq \alpha(r) \cdot a$. We call $A$ \emph{finite over $R$} (or \emph{finite as an $R$-algebra}) if it is finitely generated as an $R$-module. We call $A$ of \emph{finite type over $R$} if it is finitely generated as an $R$-algebra in the sense that there are $f_1,\ldots, f_k\in A$, $k\in \N$, such that any $R$-subalgebra $B\subseteq A$ (i.e. any subring $B\subseteq A$ which is also a $R$-submodule, or, equivalently, a subring containing the image of $\alpha$) containing the $f_i$ must equal $A$.
\end{defi}
\begin{rem}
 If $A$ is an $R$-algebra and $f_1,\ldots,f_k\in A$, the following subsets of $A$ coincide:
 \begin{itemize}
  \item $\left\{\sum_{d\in \N^k} r_d f_1^{d_1}\cdot\ldots\cdot f_k^{d_k}\middle | r_d\in R, r_d\neq 0 \text{ only for finitely many } d\right\}$
  \item The image of the ring homomorphism $R[X_1,\ldots,X_k]\longto A$ sending $p\in R[X_1,\ldots, X_k]$ to $p(f_1,\ldots,f_k)$.
  \item The intersection of all $R$-subalgebras of $A$ containing the $f_i$.
 \end{itemize}
Thus, an $R$-algebra $A$ is of finite type iff it is isomorphic to a quotient of $R[X_1,\ldots, X_k]$ by some ideal $I$ for finite $k$.
\end{rem}
\begin{rem}
\begin{enumerate}[a)]
 \item Obviously, if $f_1,\ldots, f_i\in A$ generate $A$ as an $R$-module, they generate it as an $R$-algebra. Thus any finite $R$-algebra is of finite type. On the other side, when $R\neq \{0\}$ and and $n>0$, $R[X_1, \ldots, X_n]$ is an $R$-algebra of finite type that is not finitely generated as an $R$-module.
\item Obviously, if $L/K$ is a field extension then $L$ is a finite $K$-algebra iff the field extension is finite. The fact that this still holds if $L$ is a $K$-algebra of finite type turns out to be essentially equivalent to the Nullstellensatz.
 \end{enumerate}

\end{rem}


\begin{prop}\label{prop:1.4.1}
 Let $R$ be a ring, $A$ an $R$-algebra. Any $A$-algebra $B$ becomes an $R$-algebra by composition for the homomorphisms.
 \begin{enumerate}[a)]
  \item If $A$ is finite over $R$, it is of finite type over $R$. \checkmark (trivial)
  \item (transitivity of finiteness) If $B$ is finite over $A$ and $A$ finite over $R$, then $B$ is finite over $R$.
  \item If $B$ over $A$ and $A$ over $R$ are of finite type, then $B$ is of finite type over $R$.
  \item An algebra of finite type over a Noetherian ring is a Noetherian ring.
 \end{enumerate}

\end{prop}
%end 2017-04-24
%start 2017-04-27
\begin{proof}
 \begin{enumerate}[a)]
  \item trivial
  \item If $(b_i)_{i=1}^m$ generate $B$ as an $A$-module and $(a_j)_{j=1}^n$ generate $A$ as an $R$-module, the $\beta_{i,j} = a_j\cdot b_i$ generate $B$ as an $R$-module: Let $b\in B$, then $b = \sum_{i=1}^m \alpha_i b_i$ (with $\alpha_i\in A$) and each $\alpha_i$ can be written as $\alpha_i = \sum_{j=1}^n r_{i,j}a_j$ then $b = \sum_{i=1}^m \sum_{j=1}^n r_{i,j} \beta_{i,j}$.
  \item Let $(b_i)_{i=1}^m$ generate $B$ as an $A$-module and $(a_j)_{j=1}^n$ generate $A$ as an $R$-module, then $B$ is generated by $(a_1,\ldots, a_n, b_1,\ldots, b_m)$ as an $R$-algebra. Let $\beta\in B$, then $\beta = P(b_1\ldotspam b_m) = \sum_{\alpha\in\N^m} p_\alpha b_1^{\alpha_1}\cdot\ldots\cdot b_m^{\alpha_m}$ with $p_\alpha \in A$ which can be written $p_\alpha = q_\alpha(a_1\ldotspam a_n)$ with $q_\alpha \in  R[X_1\ldotspam X_n]$, $q_\alpha= \sum_{\gamma\in \N^n} q_{\alpha,\beta}  a_1^{\gamma_1}\cdot\ldots\cdot a_n^{\gamma_n}$. Let 
  \begin{align*}
  r(X_1\ldotspam X_m, Y_1\ldotspam Y_n) = \sum_{(\alpha,\gamma)\in \N^{m+n}} q_{\alpha,\gamma} X_1^{\alpha_1}\cdot\ldots\cdot X_m^{\alpha_m}\cdot Y_1^{\gamma_1}\cdot\ldots\cdot Y_n^{\gamma_n},
  \end{align*}
 then $R(b_1\ldotspam b_m, a_1\ldotspam a_n) = \beta$ establishing our claim that $\{a_j\}\cup\{b_i\}$ generate B as an $R$-algebra.
  \item Note that the quotient of a Noetherian ring by an ideal stays Noetherian: The preimage of an infinitely ascending chain of ideals of the quotient ring would be an infinitely ascending chain of ideals of the original ring. Now if $a_1\ldotspam a_m\in A$ generate $A$ as an $R$-algebra, then
  \begin{align*}
   R[X_1\ldotspam X_m] &\longto A\\
   P&\longmapsto P(a_1\ldotspam a_m)
  \end{align*}
  is surjective and $A$ is isomorphic to a quotient of $R[X_1\ldotspam X_m]$, which by the Basissatz is Noetherian if $R$ is.
 \end{enumerate}
 \end{proof}

 \begin{prop}[Artin-Tate]\label{prop:artinTate}
  Let $R$ be a Noetherian ring, $A$ an $R$-algebra of finite type and $B\subseteq A$ an $R$-subalgebra such that $A$ is finite over $B$. Then $B$ is an $R$-algebra of finite type.
 \end{prop}
 \begin{proof}
 Let $(a_i)_{i=1}^n$ generate $A$ as an $R$-algebra and let $(\alpha_j)_{j=1}^n$ generate it as a $B$-module. We have expressions
 \begin{align}
  a_i &=\sum_{j=1}^n b_{i,j} \alpha_j\label{eq:a}\\
  \alpha_k\cdot \alpha_k &= \sum_{j=1}^n \beta_{j,k,l} \alpha_j.\label{eq:b}
 \end{align}
Let $\tilde B\subseteq B$ be the $R$-algebra generated by the $b_{i,j}$ and the $\beta_{j,k,l}$. It is of finite type over $R$ thus Noetherian. Let $\tilde A\subseteq A$ be the $\tilde B$-submodule generated by the $(\alpha_k)_{k=1}^n$. It is a subring by (\ref{eq:b}) and contains the $a_i$ by (\ref{eq:a}) and is an $R$-algebra because $\tilde B$ is. Then $\tilde A=A$ and $A$ is finite over $\tilde B$. Since $\tilde B$ is Noetherian and $B\subseteq A$ is a $\tilde B$-subalgebra and $B$ is finitely generated as $\tilde B$-module ($\tilde B$ being Noetherian), hence $B$ is of finite type over $\tilde B$ (Proposition \ref{prop:1.4.1}a), hence $B$ is of finite type over $R$ (Proposition \ref{prop:1.4.1}c)
\end{proof}
\begin{prop}[Eakin-Nagata]\label{prop:eakinNagata}
 Let $A$ be a Noetherian ring and $B\subseteq A$ be a subring such that $A$ is finite over $B$. Then $B$ is Noetherian.
\end{prop}
\begin{rem}
 See Matsumura, CRT, for Eakin-Nagata.
\end{rem}
\subsection{The notion of integrity and the Noether Normalisation Theorem}
Remark of the author: It's called integrity not entireness...
\begin{defi}\label{def:integrity}
 Let $A\subseteq B$ be a ring extension. We call $b\in B$ integral/ganz over $A$ if it satisfies an equations
 \begin{align*}
  b^n +\sum_{i=0}^{n-1} a_kb^k =0
 \end{align*}
 with $a_k\in A$. We call $B$ over $A$ integral, if every element of $B$ is integral.
\end{defi}
\begin{rem}
 It is not really necessary to assume $A\to B$ to be injective.
\end{rem}
\begin{prop}
 \begin{enumerate}[a)]
  \item $b\in B$ is integral over $A$ iff there is an intermediate ring $A\subseteq C\subseteq B$ containing $b$ which is finite over $A$. If $b_1\ldotspam b_n$ are finitely many elements of $B$ which are integral over $A$, the there is an $A$-subalgebra $A\subseteq C\subseteq B$ which is finite over $A$ and containing all $b_i$.
  \item The elements of $B$ which are integral over $A$ form a subring of $B$, the integral closure of $A$ in $B$.
  \item If $C/B$ and $B/A$ are integral, $C/A$ is integral.
  \item Let $B/A$ be integral (where it is essential that $A$ is a subring of $B$). If $B$ is a field, then $A$ is a field.
 \end{enumerate}

\end{prop}
\begin{proof}
 \begin{alphanumerate}
  \item Let $b_1\ldotspam b_n$ be integral over $A$ and let $C$ be the subring generated over $A$ by $b_1^{\alpha_1}\cdot\ldots\cdot b_n^{\alpha_n}$ with $\alpha\in \N^n$. Each $b_i$ satisfies an equation $b_i^{D_i} = \sum_{j=0}^{D_i-1} a_{i,j}\cdot b_i^j$ with $a_{i,j}\in A$. Then it follows by induction on $k$ that $b_i^k$ is an $A$-linear combination of $b_i^j$ with $0\leq j < D_i$. If follows that $C$ is generated as an $A$-module by $\left\{\prod_{i=1}^n b_i^{e_i}\middle| 0\leq e_i< D_i\right\}$ and $C$ is as desired. This the second assertion of $a)$, which contains one direction of the first as a special case. For the other direction let $C\subseteq B$ be an $A$-subalgebra which is finitely generated, e.g. by $(\gamma_i)_{i=1}^n$, as an $A$-module. Let $b\in C$, $b\gamma_i=\sum_{i=1}^n m_{j,i} \gamma_j$ with $m_{j,i}\in A$. The matrix $M=(m_{i,j})_{i=1}^n\ _{j=1}^n$ satisfies its own characteristic equation by Cayley-Hamilton: $M^n = \sum_{i=0}^{n-1}p_i M^i$ with $p_i\in A$. Since $b^j$ in $C$ can be expressed by (in the sense hat [insert diagramm here]) it follows, that $b^n \cdot c = \sum_{i=0}^{n-1} p_ib^ic$ (first for $c=\gamma_i$, then all of $C$). Taking $c=1$ shows $b^n = \sum_{i=0}^{n-1}p_i b^i$ as stated.
  \item If $C$ is as in $A$ and contains $b_1, b_2$, then it contains $b_1\pm b_2$ and $b_1\cdot b_2$, showing that these are integral over $A$. 
  \item Let, more generally, $B/A$ be integral and $c\in C$ integral over $B$. It satisfies an equation $c^d = \sum_{i=0}^{d-1} \beta_i c^i$ with $\beta_i\in B$. By a), there is an $A$-subalgebra $\tilde B\subseteq B$ which is finite over $A$ and contains the $\beta_i$. Then $c$ is integral over $\tilde B$, hence by a) there is a $\tilde B$-subalgebra $\tilde C\subseteq C$ containing $c$ and finite over $\tilde B$. Now $\tilde C/A$ is finite by Proposition \ref{prop:1.4.1}b), hence $c$ is integral over $A$ by a).
 \end{alphanumerate}


\end{proof}
%end 2017-04-27



\end{document}
